% TUGboat class documentation is at:
%   texdoc tugboat
% or
%   https://texdoc.org/serve/tugboat/0
% or
%   https://mirrors.ctan.org/macros/latex/contrib/tugboat/ltubguid.pdf

\documentclass{ltugboat}
\usepackage[T1]{fontenc}
\usepackage{graphicx}
\usepackage{microtype}
\usepackage[hyphens]{url}
\usepackage[hidelinks,pdfa]{hyperref}

\usepackage{minted}

\setminted{
  % style=algol_nu,
  style=bw,
  linenos,
  numbersep=4pt,
  % frame=lines,
  framesep=6pt,
  fontsize=\small,
}

\title{Moloch: Minimalist, Feature-Rich Beamer Presentations}
\shortTitle{Moloch}

\author{Johan Larsson}
\address{Department of Mathematical Sciences, University of Copenhagen\\Universitetsparken 5\\2100 Copenhagen\\Denmark}
\netaddress{jola@math.ku.dk}
\personalURL{https://jolars.co}
\ORCID{0000-0002-4029-5945}

\author{samcarter}
\EDITORnoaddress
\EDITORnonetaddress

% If possible, I would like to receive a physical copy of this TUGboat issue.
% Please use the address given in the \address field above.

\begin{document}
\maketitle

\begin{abstract}
  \titleref{Moloch} is a fork of the popular \LaTeX\ Beamer~\cite{wright:2025}
  theme \titleref{Metropolis}, which is no longer actively maintained. Moloch
  aims to fix outstanding issues with \titleref{Metropolis} while keeping the
  original design mostly intact. This article summarizes the changes made in
  \titleref{Moloch} as well as some general \TUB\ style guidelines for authors.
\end{abstract}

\section{Introduction}

I have long been an admirer and user of the excellent Metropolis theme, which
provides a clean, professional look for \LaTeX\ Beamer presentations in
addition to enabling useful features such as progress bars and (what I think
are) sane defaults for presentations.

The problem is that \titleref{Metropolis} is no longer actively
maintained\Dash{}its latest updates date six years back and over time an
increasing number of issues have cropped up. Most of them are minor and
corrigible through workarounds, but my preamble has grown increasingly
cluttered with such patches. Moreover, a number of users have requested
features, while others have offered contributions to help improve the theme.
Metropolis being unmaintained, however, these contributions have not been
incorporated.

This has led me to fork \titleref{Metropolis} into a new theme, which I call
\titleref{Moloch}\footnote{Likely familiar to your if you know your
  Metropolis.}. The goal is to keep the original design philosophy and look of
Metropolis while fixing outstanding issues and incorporating useful features
and contributions from the community. I have made some design changes, but
these have largely been made to make the design more accessible and to bring
the behavior of the theme closer to standard Beamer behavior. Finally, I have
also made a number of internal code improvements to make the theme more robust
and maintainable going forward. Users of \titleref{Metropolis} should find it
easy to switch to \titleref{Moloch}, whilst at the same time benefit from the
bug fixes, new features, and improved stability.

Since neither \titleref{Metropolis} nor \titleref{Moloch} have been introduced
in \TUB\ before, I will begin to do just that by introducing \titleref{Moloch}.
Users of \titleref{Metropolis} may wish to skip ahead to
Section~\ref{sec:metro-vs-moloch}, which summarizes the changes made in
\titleref{Moloch}.

\section{Getting Started with Moloch}

Moloch is a standard Beamer theme and is enabled as any other theme. Here is a
minimal example presentation using Moloch:

\begin{minted}[]{latex}
\documentclass{beamer}

\usetheme{moloch}

\title{Moloch Theme}
\subtitle{A Minimalist Beamer Theme}
\author{Johan Larsson}
\institute{
  Department of Mathematical Sciences,
  University of Copenhagen
}
\date{\today}
\titlegraphic{\hfill\includegraphics{logo.pdf}}

\begin{document}

\maketitle

\begin{frame}
  \frametitle{A Sample Frame}

  Hello, World!
\end{frame}

\end{document}
\end{minted}

This will produce a presentation with the title slide shown in
Figure~\ref{fig:title-slide}. As you can see, \titleref{Moloch} provides a
distinctive title slide with somewhat different formatting than the standard
\titleref{beamer} title slide\footnote{But if you prefer the default
  \titleref{beamer} one, then \titleref{Molochf} has the option to use that
  instead}.

\begin{figure}[htb]
  \centering
  \frame{\includegraphics[page=1,width=\columnwidth]{moloch-demo}}

  \vspace{1ex}

  \frame{\includegraphics[page=2,width=\columnwidth]{moloch-demo}}
  \caption{%
    The title slide and first frame of a presentation made with Moloch.
  }
  \label{fig:title-slide}
\end{figure}

Unlike most themes, \titleref{Moloch} comes with a number of options that can
be set either when loading the theme or via the macro \cs{molochset}. The
benefit of using the latter is that you can change options midway through your
presentations and limit them to the current scope. The following two calls are
therefore two equivalent ways to modify the title page style.

\begin{minted}{latex}
\usetheme[titlepage=plain]{moloch}
\molochset{titlepage=plain}
\end{minted}

Like other \titleref{beamer} themes, Moloch is split into a color theme, font
theme, and inner and outer themes. Here, we assume that the full Moloch theme
is loaded via \verb|\usetheme{moloch}|.

\subsection{Color Themes}

A major new feature of \titleref{Moloch} is a full-fledged color theme system
that allows users to customize the colors used in the theme in an intuitive
way. It also supports the notion of light and dark modes, which can be switched
between at any time in the presentation. In addition, we also support a number
of pre-defined color themes.

To set the overall color theme, use the option \texttt{colortheme} and
\texttt{colortheme variant} for selecting light or dark mode. For example,

\begin{minted}{latex}
\molochset{
  colortheme=catppuccin,
  colortheme variant=dark
}

\begin{frame}
  \frametitle{The Catppuccin Color Theme}

  \begin{exampleblock}{This Frame}
    This frame uses the \alert{catppuccin}
    color theme in dark mode.
  \end{exampleblock}

  Colors rely on standard color definitions,
  like
  \begin{itemize}
    \item \textcolor{mAlertFg}{
      \texttt{mAlertFg}}
    \item \textcolor{mExampleFg}{
      \texttt{mExampleFg}}
  \end{itemize}
  which allows users to tap into the current
  color theme's colors.

  \medskip

\end{frame}
\end{minted}

\begin{figure}[htpb]
  \centering
  \frame{\includegraphics[page=3,width=\columnwidth]{moloch-demo}}
  \caption{%
    A frame using the \texttt{catppuccin} color theme in dark mode.
  }
  \label{fig:colortheme-catppuccin}
\end{figure}

\section{Changes in Moloch compared to Metropolis}

\section{Documentation}

\label{sec:metro-vs-moloch}

\bibliographystyle{tugboat}
\bibliography{bibliography}

\makesignature
\end{document}
