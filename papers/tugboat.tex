% TUGboat class documentation is at:
%   texdoc tugboat
% or
%   https://texdoc.org/serve/tugboat/0
% or
%   https://mirrors.ctan.org/macros/latex/contrib/tugboat/ltubguid.pdf

\documentclass{ltugboat}
\usepackage[T1]{fontenc}
\usepackage{graphicx}
\usepackage{microtype}
\usepackage[hyphens]{url}
\usepackage[hidelinks,pdfa]{hyperref}
\usepackage{minted}
\usepackage{xcolor}

\definecolor{metropolisGreen}{HTML}{14B03D}
\definecolor{molochGreen}{RGB}{0,128,128}

\def\url{\tburl}

\setminted{
  % style=algol_nu,
  style=bw,
  linenos,
  numbersep=4pt,
  escapeinside=||,
  framesep=6pt,
  fontsize=\small,
}

\newmintinline{latex}{fontsize=\normalsize}

\title{Moloch: Minimalist, Feature-Rich Beamer Presentations}
\shortTitle{Moloch}

\author{Johan Larsson}
\address{Department of Mathematical Sciences, University of Copenhagen\\Universitetsparken 5\\2100 Copenhagen\\Denmark}
\netaddress{jola@math.ku.dk}
\personalURL{https://jolars.co}
\ORCID{0000-0002-4029-5945}

\author{samcarter}
\EDITORnoaddress
\EDITORnonetaddress

% If possible, I would like to receive a physical copy of this TUGboat issue.
% Please use the address given in the \address field above.

\begin{document}
\maketitle

\begin{abstract}
  \texttt{Moloch} is a fork of the popular \LaTeX\ \texttt{beamer} theme
  \texttt{Metropolis}. It fixes outstanding issues in \texttt{Metropolis} while
  preserving the original minimalist design. In this article we introduce
  \texttt{Moloch}'s features and options, provide guidance for
  \texttt{Metropolis} users transitioning to \texttt{Moloch}, and describe the
  \texttt{Quarto}-based documentation system that we use to generate both \PDF\
  and \HTML\ documentation.
\end{abstract}

\section{Introduction}

The excellent \texttt{Metropolis} theme~\cite{vogelgesang:2017} provides a
clean, professional look for \LaTeX\ \texttt{beamer}~\cite{samcarter:2025}
presentations as well as useful features such as progress bars and (what we
think are) sane defaults for styling. The problem is that \texttt{Metropolis}
is no longer actively maintained\Dash{}its latest updates date six years back.
Internally, it patches or redefines many internal macros and, with the
evolution of \texttt{beamer} and the \LaTeX\ kernel, these modifications have
become brittle. Over time a number of issues have cropped up. Most of these are
minor and corrigible through workarounds, but have caused users' preambles to
grow increasingly cluttered with such patches. Some users have also requested
features, while others have offered contributions to help improve the theme.
\texttt{Metropolis} being unmaintained, however, these requests and
contributions have gone unaddressed.

This has led us to fork \texttt{Metropolis} into a new theme, which we call
\texttt{Moloch}\footnote{Perhaps familiar to you if you know your Metropolis.}.
The goal is to keep the original design philosophy and look of
\texttt{Metropolis} while fixing outstanding issues and incorporating useful
features and contributions from the community. We have modified the design
slightly, but mostly with the intent of improving accessibility or to bring the
theme closer to standard \texttt{beamer} behavior. We have also improved the
internal code to make the theme robust and maintainable going forward.
In particular, \texttt{Moloch} now uses the interfaces provided by
\texttt{beamer} wherever possible, instead of patching internal macros.
Users of
\texttt{Metropolis} should find it easy to switch to \texttt{Moloch}, whilst at
the same time benefit from the bug fixes, new features, and improved stability.

Since neither \texttt{Metropolis} nor \texttt{Moloch} have been introduced in
the \TUB\ before, we will begin by introducing \texttt{Moloch} in the next
section. Users of \texttt{Metropolis} may want to skip ahead to
Section~\ref{sec:metro-vs-moloch}, which summarizes the changes made in
\texttt{Moloch} and offers guidance for transitioning from \texttt{Metropolis}
to \texttt{Moloch}. Finally, in Section~\ref{sec:documentation}, we describe
the documentation system used for \texttt{Moloch}, which generates both \PDF\
and \HTML\ documentation from a single source.

\section{Getting Started with Moloch}

Moloch is a standard \texttt{beamer} theme and is enabled in the usual way via
\latexinline{\usetheme{moloch}}. Here is a minimal example presentation using
\texttt{Moloch}:

\begin{minted}[]{latex}
\documentclass{beamer}
\usetheme{moloch}

\title{Moloch Theme}
\subtitle{A Minimalist Beamer Theme}
\author{Johan Larsson}
\institute{
  Department of Mathematical Sciences,
  University of Copenhagen
}
\date{\today}
\titlegraphic{\includegraphics{logo}}

\begin{document}

\maketitle

\begin{frame}
  \frametitle{A Sample Frame}
  Hello, World!
\end{frame}

\end{document}
\end{minted}

This produces the presentation shown in Figure~\ref{fig:title-slide}. As you
can see, \texttt{Moloch} provides a distinctive title slide with somewhat
different formatting than the standard \texttt{beamer} title slide\footnote{But
  if you prefer the default \texttt{beamer} one, then \texttt{Moloch} has the
  option to use that instead.}.

\begin{figure}[htb]
  \centering
  \frame{\includegraphics[page=1,width=\columnwidth]{moloch-demo}}

  \vspace{1ex}

  \frame{\includegraphics[page=2,width=\columnwidth]{moloch-demo}}

  \caption{%
    The title slide and first frame of a presentation made with
    Moloch.}\label{fig:title-slide}
\end{figure}

Unlike most themes, \texttt{Moloch} comes with a number of options that can be
set either when loading the theme or via the macro \cs{molochset}. The benefit
of using the latter is that you can change options midway through your
presentations and limit them to the current scope. The following two calls are
therefore two equivalent ways to modify the title page style.

\begin{minted}{latex}
\usetheme[titlepage=plain]{moloch}
\molochset{titlepage=plain}
\end{minted}

Like many other \texttt{beamer} themes, \texttt{Moloch} is split into a color theme,
font theme, and inner and outer themes. Here, we assume that the full \texttt{Moloch}
theme is loaded via \verb|\usetheme{moloch}|.

\subsection{Color Theming}

A major new feature of \texttt{Moloch} is a full-fledged color theme system
that allows users to customize the colors used in the theme in an intuitive
way. It also supports the notion of light and dark modes, which can be switched
between at any time in the presentation. In addition, we also support a number
of pre-defined color themes.

To set the overall color theme, use the option \texttt{colortheme} and
\texttt{colortheme variant} for selecting light or dark mode. For example, the
following code snippet sets the color theme to
\texttt{catppuccin}~\cite{catppuccin2026} in dark
mode~(Figure~\ref{fig:colortheme-catppuccin}).

\begin{minted}{latex}
\molochset{
  colortheme=catppuccin,
  colortheme variant=dark
}

\begin{frame}
  \frametitle{The Catppuccin Color Theme}

  \begin{exampleblock}{This Frame}
    This frame uses the \alert{catppuccin}
    color theme in dark mode.
  \end{exampleblock}

  Colors rely on standard color definitions,
  like
  \begin{itemize}
    \item \textcolor{mAlertFg}{
      \texttt{mAlertFg}}
    \item \textcolor{mExampleFg}{
      \texttt{mExampleFg}}
  \end{itemize}
  which allows users to tap into the current
  color theme's colors.
\end{frame}
\end{minted}

\begin{figure}[htb]
  \centering
  \frame{\includegraphics[page=3,width=\columnwidth]{moloch-demo}}

  \caption{%
    A frame using the \texttt{catppuccin} color theme in dark mode.}
  \label{fig:colortheme-catppuccin}
\end{figure}

For more granular color customization, \texttt{Moloch} features a new macro,
\cs{molochcolors}, which sets individual color elements of the theme. Users
already familiar with \texttt{beamer}'s existing options for color
customization (e.g.\ \cs{setbeamercolor}) might wonder what this macro brings to
the table. The answer is that it allows users to separately specify colors for
light and dark modes of the theme. The general syntax for this is
\latexinline{\molochcolors{|\meta{mode}|/|\meta{element}|=|\meta{color}|}},
where \meta{mode} is either \tbcode{light} or \tbcode{dark}, \meta{element} is
the name of the color element to set, and \meta{color} is the color
specification.

For instance, the following code snippet customizes the frame title foreground
and background colors, the progress bar color, and the alerted and example text
colors in light mode~(Figure~\ref{fig:custom-colors}).

\begin{minted}[]{latex}
\molochcolors{
  light/frametitle fg=black,
  light/frametitle bg=yellow!20,
  light/progressbar fg=magenta!90,
  light/alerted text=red!80!black,
  light/example text=blue!60,
}
\end{minted}

\begin{figure}[htb]
  \frame{\includegraphics[page=10,width=\columnwidth]{moloch-demo}}

  \caption{%
    A frame demonstrating customized colors in light
    mode.}\label{fig:custom-colors}
\end{figure}

In addition, omitting the \meta{mode} part will set the color for the current
mode. For example, the following code snippet sets the color of example blocks
to green in the current mode.

\begin{minted}{latex}
\molochcolors{
  example fg=green!70!black
}
\end{minted}

The end result is that \texttt{Moloch} provides a flexible and (we hope)
intuitive color theming system that allows users to easily customize the look
of their presentations and effortlessly switch between light and dark modes at
their convenience.

To enable this color system functionality, \texttt{Moloch} makes heavy use of
the facilities of \texttt{pgfkeys}~\cite{tantau:2007} from
\acro{PGF}/\texttt{TikZ}~\cite{feuersanger:2025}.

\subsection{Progress Bars}

A popular feature of \texttt{Metropolis} (that \texttt{Moloch} retains) is the
ability to track progress through the slides via a progress bar on either the
frames or on section and/or subsection pages. To setup progress bars for
frames, use the option \texttt{progressbar} with either \texttt{head},
\texttt{foot}, or \texttt{frametitle} to place the progress bar in the header,
footer, or below the frame title, respectively. The following code snippet
places a thick progress bar in the footer of each
frame~(Figure~\ref{fig:progress-bars}).

\begin{minted}{latex}
\molochset{
  progressbar=foot,
  progressbar linewidth=4pt
}
\end{minted}

\begin{figure*}[htb]
  \centering
  \frame{\includegraphics[page=4,width=0.32\textwidth]{moloch-demo}}\hfill
  \frame{\includegraphics[page=5,width=0.32\textwidth]{moloch-demo}}\hfill
  \frame{\includegraphics[page=6,width=0.32\textwidth]{moloch-demo}}

  \caption{%
    Progress bars in the footer, header, and below the frame title. Note that we
    have increased the line width of the bars to make them more visible. In the
    default setting, the bars are less obtrusive.}\label{fig:progress-bars}
\end{figure*}

Unlike \texttt{beamer}, \texttt{Moloch} comes with section (but not subsection)
pages enabled by default, which means that sections (using \cs{section}) will
result in dedicated frames. By default, they use the same progress bar style as
frames~(Figure~\ref{fig:section-page-default}), but this can be changed via
\latexinline{\molochset{sectionpage=|\meta{option}|}}. The available options
are \tbcode{progressbar} (the default), \tbcode{none} (no section page) and
\tbcode{simple} (a section page without a progress bar).

\begin{figure}[htb]
  \centering
  \frame{\includegraphics[page=7,width=0.49\columnwidth]{moloch-demo}}
  \hfill
  \frame{\includegraphics[page=8,width=0.49\columnwidth]{moloch-demo}}

  \caption{%
    The default (left) and simple (right) section pages styles. The section page
    can also be disabled entirely via
    \latexinline[fontsize=\small]{\molochset{sectionpage=none}}.}\label{fig:section-page-default}
\end{figure}

\subsection{Standout Frames}

\texttt{Moloch} (as well as \texttt{Metropolis}) supports special
\emph{standout} frames, which are intended as a means for highlighting
important frames in the presentation. They are created by specifying a
\tbcode{standout} key to the \tbcode{frame} environment. For example, the
following code snippet creates the standout frame seen in
Figure~\ref{fig:standout-frame}.

\begin{minted}{latex}
\begin{frame}[standout]
  Thank you!
\end{frame}
\end{minted}

\begin{figure}[tpb]
  \centering
  \includegraphics[page=9,width=\columnwidth]{moloch-demo}

  \caption{%
    A standout frame created via \latexinline[fontsize=\small]{\begin{frame}[standout] ... \end{frame}}.}\label{fig:standout-frame}
\end{figure}

As with most elements of \texttt{Moloch}, the colors of standout frames can be
customized via the color theme system.

\subsection{Title Page Styles}

As we have already seen, \texttt{Moloch} provides a distinctive title page
style by default. As usual, however, preferences differ and for that reason,
\texttt{Moloch} comes with three different title page styles that can be
selected via the \tbcode{titlepage} option. The available styles are
\tbcode{moloch} (the default), \tbcode{plain} (a centered, \texttt{beamer}-like
style), and \tbcode{split} which splits the information by a vertical
line~(Figure~\ref{fig:title-page-styles}).

\begin{figure*}[htpb]
  \centering
  \frame{\includegraphics[page=1,width=0.32\textwidth]{moloch-demo}}\hfill
  \frame{\includegraphics[page=11,width=0.32\textwidth]{moloch-demo}}\hfill
  \frame{\includegraphics[page=12,width=0.32\textwidth]{moloch-demo}}

  \caption{%
    The three title page styles: \texttt{moloch} (left), \texttt{plain} (center),
    and \texttt{split} (right).}\label{fig:title-page-styles}
\end{figure*}

\section{Transitioning from Metropolis to Moloch}
\label{sec:metro-vs-moloch}

\texttt{Moloch} retains much of the original design and behavior of
\texttt{Metropolis}, but there \emph{are} differences. In this section we
summarize these, explain their motivation, and help users transition from
\texttt{Metropolis} to \texttt{Moloch}. Although many of the changes in
\texttt{Moloch} are internal code improvements and mitigations of bugs, we will
not focus on these here and instead highlight changes that will be visible to
users.

One important difference is that we have tried to make \texttt{Moloch} more
consistent with standard \texttt{beamer} behavior, which makes it easier for
\texttt{beamer} users to adopt \texttt{Moloch}\Dash{}at the cost of requiring
\texttt{Metropolis} users to adapt. A guiding principle has been to not
re-invent the wheel when \texttt{beamer} already provides the desired
functionality. This has led to some changes in behavior compared to
\texttt{Metropolis}, as we now, for instance, prefer to rely on
\texttt{beamer}'s built-in mechanisms for frame numbering and definitions of
block environments. Another related change is to avoid redefining internal
\texttt{beamer} macros. This was a source of brittleness in \texttt{Metropolis}
that forced upstream workarounds in \texttt{beamer} to avoid breaking changes.
By relying on standard \texttt{beamer} mechanisms instead, \texttt{Moloch} is
more robust and future-proof.

\subsection{New Example Color}

While we generally find \texttt{Metropolis} well-designed, we believe that the
color used in example blocks can be difficult to read for some users,
especially those with color vision deficiencies. Therefore, we have changed
from the lighter and vivid
\textcolor{metropolisGreen}{green}~(\textcolor{metropolisGreen}{\rule{1.2ex}{1.2ex}})
in \texttt{Metropolis} to a darker, but less saturated
\textcolor{molochGreen}{teal}~(\textcolor{molochGreen}{\rule{1.2ex}{1.2ex}}) in
\texttt{Moloch}. Users who prefer the original color can revert via the
following snippet.

\begin{minted}{latex}
\definecolor{metropolisGreen}{HTML}{14B03D}
\molochcolors{example fg=metropolisGreen}
\end{minted}

\subsection{Setting Options}

\texttt{Metropolis} introduces \cs{metroset} for setting theme options.
\texttt{Moloch} instead uses \cs{molochset}, which is otherwise used exactly in
the same way.

\subsection{Typography}

\texttt{Metropolis} features a preference for \texttt{Fira Sans} (designed by
Erik Spiekermann). For Linux users compiling with \LuaLaTeX\ and \XeLaTeX, the
package automatically tries to find, configure, and use this font. We
appreciate the design of \texttt{Fira Sans}, but also feel that the font choice
should be left to the user, particularly since the behavior in
\texttt{Metropolis} leads different users to see different fonts depending on
their system setup.

In \texttt{Moloch}, we have therefore removed the automatic font selection and
instead leave the choice of font to the user. Those who still wish to use
\texttt{Fira Sans} can do so by adding the following lines to their preamble.

\begin{minted}[]{latex}
\usepackage[medium,light]{FiraSans}
\usepackage{FiraMono}
\usepackage{firamath-otf}
\end{minted}

To better mimic the original \texttt{Metropolis} setup, options
\tbcode{weight=Light} and \tbcode{usefilenames} can be passed to
\texttt{firamath-otf}, but note that this requires manual installation of the
appropriate weights of the \texttt{Fira Math} font.

Users of \LuaLaTeX\ and \XeLaTeX\ may instead wish to use
\texttt{fontspec}~\cite{latexproject:2025} to set up \texttt{Fira Sans} as the
main font. Here is an example configuration that matches the original
\texttt{Metropolis} setup.

\begin{minted}[]{latex}
\usepackage{fontspec}

\setsansfont[
  ItalicFont={Fira Sans Light Italic},
  BoldFont={Fira Sans},
  BoldItalicFont={Fira Sans Italic}
]{Fira Sans Light}

\setmonofont[
  BoldFont={Fira Mono Medium}
]{Fira Mono}

\setmathfont[
  Scale=MatchLowercase,
  ItalicFont={Fira Math Italic},
  BoldFont={Fira Math Bold}
]{Fira Math}

\AtBeginEnvironment{tabular}{%
  \addfontfeature{Numbers={Monospaced}}
}
\end{minted}

\subsection{Paragraph Spacing and Line Spacing}

Unlike standard \texttt{beamer}, in which \cs{parskip} is set to zero,
\texttt{Metropolis} instead sets it to 0.5~em units. This means that in
\texttt{Metropolis}, you do not need to use manual paragraph spacing commands
like \cs{medskip} to separate paragraphs.

But this had some undesired side-effects, such as introducing additional
spacing inside block environments, around captions, and list environments. It
also meant that users could not easily switch from \texttt{Metropolis} to
standard \texttt{beamer} without having to adjust their content to accommodate
the different paragraph spacing. For those reasons, we have removed this
setting in \texttt{Moloch}. That means that users transitioning will need to
either separate paragraphs manually (e.g. via \cs{medskip}) or use
\latexinline{\setlength{\parskip}{0.5em}} to reinstate the original behavior.

\texttt{Metropolis} also increases line spacing compared to standard
\texttt{beamer}, which we have also removed in \texttt{Moloch} for similar
reasons and to improve consistency with standard \texttt{beamer}. Users who
wish to reinstate this behavior can do so via \latexinline{\linespread{1.15}}.

Note, however, that changing the paragraph or line spacing may affect the
overall layout of the slides and you may need to adjust other elements
accordingly.

\subsection{PGFPlots Theme}

\texttt{Metropolis} comes with a custom \texttt{pgfplots}~\cite{pgfplots} theme
that uses colors from Paul Tol's 21-color theme~\cite{carl2013}. We removed
this theme from \texttt{Moloch} since we felt that it was outside the scope of
a \texttt{beamer} theme to provide a plotting theme. The theme is still
provided through \texttt{Metropolis}, however, so users who wish to use it can
do so by adding the following line to their preamble.

\begin{minted}[]{latex}
\usepackage{pgfplots,pgfplotsthemetol}
\end{minted}

\subsection{Frame Subtitles}

\texttt{Moloch}, unlike \texttt{Metropolis}, supports frame subtitles. The
reasoning for this omission was that the author felt they were unnecessary
clutter. We generally agree that subtitles should be used sparingly (if at
all), but we also do not want to force this preference upon our users.

\subsection{Frame Numbering}

Metropolis sported its own frame numbering system, which worked fine except it
required a separate package (\texttt{appendixnumberbeamer}~\cite{lelong:2018})
to have frame numbering restart (and not count towards the total number) for
appendix slides. \texttt{beamer} now supports this natively, however, and so
\texttt{Moloch} simply uses this built-in functionality. The design is
\emph{slightly} different, with smaller font size and somewhat different
margins.

\subsection{Block Environments}

\texttt{Metropolis} supports an alternative filled block style, which uses a
different background color for block environments. Many other \texttt{beamer}
themes support filled blocks too. Unlike most of these, however,
\texttt{Metropolis} alters the alignment of the text in these filled block
environments such that the main body text (for the frame) aligns with the left
edge of the box and not the text inside the box. Users have, however, reported
problems with this approach. And because we prefer the default behavior in
\texttt{beamer} also for typographical reasons and consistency\footnote{Since
  changing the block style in \texttt{Metropolis} also risks modifying the layout
  of the slide.}, we have reverted to the standard \texttt{beamer} behavior in
\texttt{Moloch}. See Figure~\ref{fig:block-comparison} for a comparison.

\begin{figure}[htb]
  \centering
  \frame{\includegraphics[
      width=0.48\columnwidth,
      trim=0 0 170 0,
      clip=true
    ]{moloch-blocks}}\hfill
  \frame{\includegraphics[
      width=0.48\columnwidth,
      trim=0 0 170 0,
      clip=true
    ]{metropolis-blocks}}

  \caption{%
    A comparison of alignment of block environments in Moloch (left) and Metropolis
    (right). } \label{fig:block-comparison}
\end{figure}

\subsection{Itemize Lists}

\texttt{Metropolis} uses \cs{textbullet}~(\textbullet) as the symbol for all
levels of the itemize environment. The problem with \cs{textbullet} is that it
is defined differently in different fonts. Notably, for instance, in
\titleref{Latin Modern}~\cite{latinmodern} it is defined as a filled square,
which looks quite different from the round bullet used in most other fonts.

The appearance of itemize symbols is a frequent source of issues because they
typically depend on the font. Another common setting is to use a math symbol,
like \latexinline{$\bullet$}~($\bullet$), but then the problem is only shifted
to depend on the math font instead.

For these reasons, \texttt{Moloch} uses \acro{PGF} primitives to draw all of
its itemize symbols, which means that they work regardless of the font used.
Also, to improve the visual hierarchy, \texttt{Moloch} uses different symbols
for different levels of the itemize
environment~(Figure~\ref{fig:itemize-comparison}). Users who nevertheless wish
to switch to the \texttt{Metropolis} symbol can simply specify the following in
their preamble.

\begin{minted}[]{latex}
\setbeamertemplate{itemize items}{\textbullet}
\end{minted}

\begin{figure}[htb]
  \centering
  \includegraphics[
    width=0.48\columnwidth,
    trim=0 180 200 0,
    clip=true
  ]{lists-moloch}
  \includegraphics[
    width=0.48\columnwidth,
    trim=0 180 200 0,
    clip=true
  ]{lists-metropolis}

  \caption{%
    The design of itemize lists in \texttt{Moloch} (left) and \texttt{Metropolis}
    (right).}\label{fig:itemize-comparison}
\end{figure}

\subsection{Other Miscellaneous Changes}

\texttt{Moloch} includes many other changes. We will not go into all the
details here; but briefly, we have
\begin{itemize}
  \item redesigned the default title page style,
  \item modernized the build system to use \texttt{l3build}~\cite{mittelbach:2014} and
        include unit tests, and
  \item removed hard-coding of the bibliography style.
\end{itemize}

\section{Documentation}
\label{sec:documentation}

\LaTeX\ packages typically come with \PDF\ documentation and \texttt{Moloch} is
no exception, featuring both a user guide and implementation documentation.
Yet, although it is standard to supply documentation as a \PDF\ manual, this
format has some downsides. Notably, it is not responsive to screen
size\Dash{}being hard to read and navigate on small screen and making
suboptimal use of the screen size on large, widescreen displays. It also lacks
good search functionality, is difficult to cross-reference outside of the
document itself, and, perhaps most importantly, is less accessible than web
documentation.

\texttt{Moloch} therefore also comes with web documentation available at
\url{https://moloch.ink}. The site features \HTML-based versions of both the
user guide and implementation documentation, in similar fashion as the
unofficial documentation for \texttt{PGF}/\texttt{TikZ} at
\url{https://tikz.dev}. Yet, whereas that site is generated via
\texttt{lwarp}~\cite{dunn:2025}, \texttt{Moloch} takes a somewhat different
approach. Since we think this approach might be of interest to other package
authors, we will briefly summarize it here.

\subsection{Architecture}

The manual for the package is in fact written in \texttt{Markdown},
specifically the flavor supported by \texttt{Quarto}~\cite{allaire:2025}, which
is built on \texttt{Pandoc}~\cite{macfarlane:2025}. \texttt{Quarto} has
powerful features for turning Markdown source into both \PDF\ books and \HTML\
web sites. This allows us to provide a single source of truth for both the
\PDF\ and \HTML\ documentation.

One challenge is that we prefer the literate programming style of
\texttt{docstrip}~\cite{mittelbach:2025} and therefore document the
implementation using standard \LaTeX, which means we maintain documentation in
both \texttt{Markdown} and \LaTeX. Fortunately, \texttt{Pandoc} supports
conversion both to \emph{and} from \LaTeX, which allows us to convert the
documentation from the \texttt{.dtx} format to Markdown after stripping it
through a custom script.

This then gives us a complete \texttt{Markdown} source file for the entire
documentation, which we then convert to both a \PDF\ manual~(through \LaTeX)
and an \HTML\ website via \texttt{Quarto}. In order to provide proper styling
for documentation of macros and package options, we use \tbcode{Lua} filters,
which is a built-in feature of \texttt{Quarto}, to keep a unified syntax for
the documentation. Here is, for instance, an example of a documentation block
for the \tbcode{block} option in \texttt{Moloch}.

\begin{minted}[]{markdown}
::: {.describe-option 
     option-name="block" 
     values="transparent, fill" 
     default="transparent"
}
Optionally adds a light grey background to block
environments like `theorem` and `example`.
:::
\end{minted}

The output of this block is shown in Figure~\ref{fig:describe-option} both as
rendered in \LaTeX\ and \HTML.

\begin{figure}[htpb]
  \centering

  \LaTeX

  \smallskip

  \frame{
    \begin{minipage}{\columnwidth}%
      \vspace{3pt}\hspace{3pt}%
      \includegraphics[width=\dimexpr\columnwidth-6pt\relax]{describe-option-pdf}%
      \vspace{3pt}%
    \end{minipage}%
  }

  \medskip

  \HTML

  \smallskip

  \frame{
    \begin{minipage}{\columnwidth}%
      \vspace{3pt}\hspace{3pt}%
      \includegraphics[width=\dimexpr\columnwidth-6pt\relax]{describe-option-html}%
      \vspace{3pt}
    \end{minipage}%
  }

  \caption{%
    The same documentation block rendered in \LaTeX\ (top) and \HTML\ (bottom) via
    a custom \texttt{Quarto} filter.}\label{fig:describe-option}
\end{figure}

\subsection{Reusability}

Our documentation system is part of the internal machinery of \texttt{Moloch}
and would currently require some effort to adapt to another package. But we
hope that this section may have piqued some interest and would be happy to
collaborate to turn this into a reusable feature\Dash{}perhaps as a separate
package\Dash{}in order to help other authors provide web documentation for
their packages.

\section{Conclusion}

We have presented \texttt{Moloch}, a fork of the popular \texttt{beamer} theme
\texttt{Metropolis}. \texttt{Moloch} retains the minimalist design philosophy
of \texttt{Metropolis} while fixing outstanding issues and incorporating new
features. Our hope is that \texttt{Moloch} will serve as a robust and
maintainable alternative to \texttt{Metropolis} going forward and that users of
both ``regular'' \texttt{beamer} themes and \texttt{Metropolis} will find it
easy to adopt.

\texttt{Moloch} is developed openly at \url{https://github.com/jolars/moloch}
and its documentation is available at \url{https://moloch.ink}. We welcome
contributions and feedback from the community to help improve \texttt{Moloch}
going forward.

\section{Acknowledgments}

\bibliographystyle{tugboat}
\bibliography{bibliography}

\makesignature
\end{document}
