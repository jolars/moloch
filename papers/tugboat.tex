% TUGboat class documentation is at:
%   texdoc tugboat
% or
%   https://texdoc.org/serve/tugboat/0
% or
%   https://mirrors.ctan.org/macros/latex/contrib/tugboat/ltubguid.pdf

\documentclass{ltugboat}
\usepackage[T1]{fontenc}
\usepackage{graphicx}
\usepackage{microtype}
\usepackage[hyphens]{url}
\usepackage[hidelinks,pdfa]{hyperref}

\usepackage{minted}

\def\url{\tburl}

\setminted{
  % style=algol_nu,
  style=bw,
  linenos,
  numbersep=4pt,
  escapeinside=||,
  framesep=6pt,
  fontsize=\small,
}

\newmintinline{latex}{fontsize=\normalsize}

\title{Moloch: Minimalist, Feature-Rich Beamer Presentations}
\shortTitle{Moloch}

\author{Johan Larsson}
\address{Department of Mathematical Sciences, University of Copenhagen\\Universitetsparken 5\\2100 Copenhagen\\Denmark}
\netaddress{jola@math.ku.dk}
\personalURL{https://jolars.co}
\ORCID{0000-0002-4029-5945}

\author{samcarter}
\EDITORnoaddress
\EDITORnonetaddress

% If possible, I would like to receive a physical copy of this TUGboat issue.
% Please use the address given in the \address field above.

\begin{document}
\maketitle

\begin{abstract}
  \texttt{Moloch} is a fork of the popular \LaTeX\ \texttt{beamer} theme
  \texttt{Metropolis}, which is no longer actively maintained. Moloch aims to
  fix outstanding issues with \texttt{Metropolis} while keeping the original
  design mostly intact. This article introduces \texttt{Moloch} and summarizes
  the changes made compared to \texttt{Metropolis}.
\end{abstract}

\section{Introduction}

I have long been an admirer and user of the excellent \texttt{Metropolis}
theme~\cite{vogelgesang:2017}, which provides a clean, professional look for
\LaTeX\ \texttt{beamer}~\cite{samcarter:2025} presentations in addition to
enabling useful features such as progress bars and (what I think are) sane
defaults for presentations.

The problem is that \texttt{Metropolis} is no longer actively
maintained\Dash{}its latest updates date six years back and over time an number
of issues have cropped up. Most of them are minor and corrigible through
workarounds, but my preamble has grown increasingly cluttered with such
patches. Moreover, a number of users have requested features, while others have
offered contributions to help improve the theme. Metropolis being unmaintained,
however, these contributions have not been incorporated.

This has led me to fork \texttt{Metropolis} into a new theme, which I call
\texttt{Moloch}\footnote{Likely familiar to your if you know your Metropolis.}.
The goal is to keep the original design philosophy and look of Metropolis while
fixing outstanding issues and incorporating useful features and contributions
from the community. I have made some design changes, but these have largely
been made to make the design more accessible and to bring the behavior of the
theme closer to standard \texttt{beamer} behavior. Finally, I have also made a
number of internal code improvements to make the theme more robust and
maintainable going forward. Users of \texttt{Metropolis} should find it easy to
switch to \texttt{Moloch}, whilst at the same time benefit from the bug fixes,
new features, and improved stability.

Since neither \texttt{Metropolis} nor \texttt{Moloch} have been introduced in
\TUB\ before, I will begin to do just that by introducing \texttt{Moloch}.
Users of \texttt{Metropolis} may wish to skip ahead to
Section~\ref{sec:metro-vs-moloch}, which summarizes the changes made in
\texttt{Moloch}.

\section{Getting Started with Moloch}

Moloch is a standard \texttt{beamer} theme and is enabled as any other theme.
Here is a minimal example presentation using \texttt{Moloch}:

\begin{minted}[]{latex}
\documentclass{beamer}
\usetheme{moloch}

\title{Moloch Theme}
\subtitle{A Minimalist Beamer Theme}
\author{Johan Larsson}
\institute{
  Department of Mathematical Sciences,
  University of Copenhagen
}
\date{\today}
\titlegraphic{\includegraphics{logo.pdf}}

\begin{document}

\maketitle

\begin{frame}
  \frametitle{A Sample Frame}
  Hello, World!
\end{frame}

\end{document}
\end{minted}

This will produce a presentation with the title slide shown in
Figure~\ref{fig:title-slide}. As you can see, \texttt{Moloch} provides a
distinctive title slide with somewhat different formatting than the standard
\texttt{beamer} title slide\footnote{But if you prefer the default
  \texttt{beamer} one, then \texttt{Molochf} has the option to use that instead}.

\begin{figure}[htb]
  \centering
  \frame{\includegraphics[page=1,width=\columnwidth]{moloch-demo}}

  \vspace{1ex}

  \frame{\includegraphics[page=2,width=\columnwidth]{moloch-demo}}
  \caption{%
    The title slide and first frame of a presentation made with Moloch.
  }
  \label{fig:title-slide}
\end{figure}

Unlike most themes, \texttt{Moloch} comes with a number of options that can be
set either when loading the theme or via the macro \cs{molochset}. The benefit
of using the latter is that you can change options midway through your
presentations and limit them to the current scope. The following two calls are
therefore two equivalent ways to modify the title page style.

\begin{minted}{latex}
\usetheme[titlepage=plain]{moloch}
\molochset{titlepage=plain}
\end{minted}

Like other \texttt{beamer} themes, Moloch is split into a color theme, font
theme, and inner and outer themes. Here, we assume that the full Moloch theme
is loaded via \verb|\usetheme{moloch}|.

\subsection{Color Theming}

A major new feature of \texttt{Moloch} is a full-fledged color theme system
that allows users to customize the colors used in the theme in an intuitive
way. It also supports the notion of light and dark modes, which can be switched
between at any time in the presentation. In addition, we also support a number
of pre-defined color themes.

To set the overall color theme, use the option \texttt{colortheme} and
\texttt{colortheme variant} for selecting light or dark mode. For example, the
following code snippet sets the color theme to \texttt{catppuccin} in dark
mode~(Figure~\ref{fig:colortheme-catppuccin}).

\begin{minted}{latex}
\molochset{
  colortheme=catppuccin,
  colortheme variant=dark
}

\begin{frame}
  \frametitle{The Catppuccin Color Theme}

  \begin{exampleblock}{This Frame}
    This frame uses the \alert{catppuccin}
    color theme in dark mode.
  \end{exampleblock}

  Colors rely on standard color definitions,
  like
  \begin{itemize}
    \item \textcolor{mAlertFg}{
      \texttt{mAlertFg}}
    \item \textcolor{mExampleFg}{
      \texttt{mExampleFg}}
  \end{itemize}
  which allows users to tap into the current
  color theme's colors.
\end{frame}
\end{minted}

\begin{figure}[htb]
  \centering
  \frame{\includegraphics[page=3,width=\columnwidth]{moloch-demo}}
  \caption{%
    A frame using the \texttt{catppuccin} color theme in dark mode.
  }
  \label{fig:colortheme-catppuccin}
\end{figure}

For more granular color customization, \texttt{Moloch} features a new macro,
\cs{molochcolors}, which sets individual color elements of the theme. Users
already familiar with \texttt{beamer}'s existing options for color
customization (e.g. \cs{setbeamercolor}) might wonder what this macro brings to
the table. The answer is that it allows users to separately specify colors for
light and dark modes of the theme. The general syntax for this is
\latexinline{\molochcolors{|\meta{mode}|/|\meta{element}|=|\meta{color}|}},
where \meta{mode} is either \tbcode{light} or \tbcode{dark}, \meta{element} is
the name of the color element to set, and \meta{color} is the color
specification.

For instance, the following code snippet customizes the frame title foreground
and background colors, the progress bar color, and the alerted and example text
colors in light mode~(Figure~\ref{fig:}).

\begin{minted}[]{latex}
\molochcolors{
  light/frametitle fg=black,
  light/frametitle bg=yellow!20,
  light/progressbar fg=magenta!90,
  light/alerted text=red!80!black,
  light/example text=blue!60,
}
\end{minted}

\begin{figure}[htb]
  \centering
  \frame{\includegraphics[page=10,width=\columnwidth]{moloch-demo}}
  \caption{%
    A frame demonstrating customized colors in light mode.
  }
  \label{fig:}
\end{figure}

In addition, omitting the \meta{mode} part will set the color for the current
mode. For example, the following code snippet sets the color of example blocks
to green in the current mode.

\begin{minted}{latex}
\molochcolors{
  example fg=green!70!black
}
\end{minted}

The end result is that \texttt{Moloch} provides a flexible and (we hope)
intuitive color theming system that allows users to easily customize the look
of their presentations and effortlessly switch between light and dark modes at
their convenience.

To enable this color system functionality, \texttt{Moloch} makes heavy use of
the facilities of \texttt{pgfkeys}~\cite{tantau:2007} from
\acro{PGF}/\texttt{TikZ}~\cite{feuersanger:2025}.

\subsection{Progress Bars}

A popular feature of \texttt{Metropolis} (that \texttt{Moloch} retains) is
ability to track progress through the slides via a progress bar on either the
frames or on section and/or subsection pages. To setup progress bars for the
frames, use the option \texttt{progressbar} with one \texttt{head},
\texttt{foot}, or \texttt{frametitle} to place the progress bar in the header,
footer, or below the frame title, respectively. The following code snippet
places a thick progress bar in the footer of each
frame~(Figure~\ref{fig:progress-bars}).

\begin{minted}{latex}
\molochset{
  progressbar=foot,
  progressbar linewidth=4pt
}
\end{minted}

\begin{figure*}[htb]
  \centering
  \frame{\includegraphics[page=4,width=0.32\textwidth]{moloch-demo}}\hfill
  \frame{\includegraphics[page=5,width=0.32\textwidth]{moloch-demo}}\hfill
  \frame{\includegraphics[page=6,width=0.32\textwidth]{moloch-demo}}

  \caption{%
    Progress bars in the footer, header, and below the frame title.
    Note that we have increased the line width of the bars to make them
    more visible. In the default setting, the bars are less obtrusive.
  }
  \label{fig:progress-bars}
\end{figure*}

Unlike \texttt{beamer}, \texttt{Moloch} comes with section (but not subsection)
pages enabled by default, which means that sections (using \cs{section}) will
result in dedicated frames. By default, they use the same progress bar style as
frames~(Figure~\ref{fig:section-page-default}), but this can be set via
\latexinline{\molochset{sectionpage=|\meta{option}|}}, with \tbcode{none} and
\tbcode{simple} for no section page and a simple section page without a
progress bar, respectively.

\begin{figure}[htb]
  \centering
  \frame{\includegraphics[page=7,width=0.49\columnwidth]{moloch-demo}}
  \hfill
  \frame{\includegraphics[page=8,width=0.49\columnwidth]{moloch-demo}}

  \caption{%
    The default (left) and simple (right) section pages styles.
    The section page can also be disabled entirely via
    \latexinline{\molochset{sectionpage=none}}.
  }
  \label{fig:section-page-default}
\end{figure}

\subsection{Standout Frames}

\texttt{Moloch} (as well as \texttt{Metropolis}) supports special \emph{standout}
frames, which are intended as a mean through which to highlight important frames in
the presentation. They are created by specifying a \tbcode{standout} key to
the \tbcode{frame} environment. For example, the following code snippet
creates the standout frame seen in Figure~\ref{fig:standout-frame}.

\begin{minted}{latex}
\begin{frame}[standout]
  Thank you!
\end{frame}
\end{minted}

\begin{figure}[htpb]
  \centering
  \includegraphics[page=9,width=\columnwidth]{moloch-demo}
  \caption{%
    A standout frame created via
    \latexinline{\begin{frame}[standout] ... \end{frame}}.
  }
  \label{fig:standout-frame}
\end{figure}

As with most elements of \texttt{Moloch}, the colors of standout frames can be
customized via the color theme system.

\subsection{Title Page Styles}

As we have already seen, \texttt{Moloch} provides a distinctive title page
style by default. As usual, however, preferences differ and for that reason,
\texttt{Moloch} comes with three different title page styles that can be
selected via the \tbcode{titlepage} option. The available styles are
\tbcode{moloch} (the default), \tbcode{plain} (a centered, \texttt{beamer}-like
style), and \tbcode{split} which splits the information by a vertical
line~(Figure~\ref{fig:title-page-styles}).

\begin{figure*}[htpb]
  \centering
  \frame{\includegraphics[page=1,width=0.32\textwidth]{moloch-demo}}\hfill
  \frame{\includegraphics[page=11,width=0.32\textwidth]{moloch-demo}}\hfill
  \frame{\includegraphics[page=12,width=0.32\textwidth]{moloch-demo}}
  \caption{%
    The three title page styles: \texttt{moloch} (left), \texttt{plain} (center),
    and \texttt{split} (right).
  }
  \label{fig:title-page-styles}
\end{figure*}

\section{Changes in Moloch compared to Metropolis}
\label{sec:metro-vs-moloch}

\section{Documentation}

\section{Conclusion}

\texttt{Moloch} is developed openly on GitHub
at \url{https://github.com/jolars/moloch} and its
documentation is available at \url{https://moloch.ink}.

We welcome contributions and feedback from the community to help improve
\texttt{Moloch} going forward.

\section{Acknowledgments}

\bibliographystyle{tugboat}
\bibliography{bibliography}

\makesignature
\end{document}
