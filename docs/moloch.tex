% Options for packages loaded elsewhere
\PassOptionsToPackage{unicode}{hyperref}
\PassOptionsToPackage{hyphens}{url}
\PassOptionsToPackage{dvipsnames,svgnames,x11names}{xcolor}
%
\documentclass[
  letterpaper,
  numbers=noendperiod,
  DIV=10]{scrreprt}

\usepackage{amsmath,amssymb}
\usepackage{iftex}
\ifPDFTeX
  \usepackage[T1]{fontenc}
  \usepackage[utf8]{inputenc}
  \usepackage{textcomp} % provide euro and other symbols
\else % if luatex or xetex
  \usepackage{unicode-math}
  \defaultfontfeatures{Scale=MatchLowercase}
  \defaultfontfeatures[\rmfamily]{Ligatures=TeX,Scale=1}
\fi
\usepackage{lmodern}
\ifPDFTeX\else  
    % xetex/luatex font selection
\fi
% Use upquote if available, for straight quotes in verbatim environments
\IfFileExists{upquote.sty}{\usepackage{upquote}}{}
\IfFileExists{microtype.sty}{% use microtype if available
  \usepackage[]{microtype}
  \UseMicrotypeSet[protrusion]{basicmath} % disable protrusion for tt fonts
}{}
\makeatletter
\@ifundefined{KOMAClassName}{% if non-KOMA class
  \IfFileExists{parskip.sty}{%
    \usepackage{parskip}
  }{% else
    \setlength{\parindent}{0pt}
    \setlength{\parskip}{6pt plus 2pt minus 1pt}}
}{% if KOMA class
  \KOMAoptions{parskip=half}}
\makeatother
\usepackage{xcolor}
\setlength{\emergencystretch}{3em} % prevent overfull lines
\setcounter{secnumdepth}{5}
% Make \paragraph and \subparagraph free-standing
\makeatletter
\ifx\paragraph\undefined\else
  \let\oldparagraph\paragraph
  \renewcommand{\paragraph}{
    \@ifstar
      \xxxParagraphStar
      \xxxParagraphNoStar
  }
  \newcommand{\xxxParagraphStar}[1]{\oldparagraph*{#1}\mbox{}}
  \newcommand{\xxxParagraphNoStar}[1]{\oldparagraph{#1}\mbox{}}
\fi
\ifx\subparagraph\undefined\else
  \let\oldsubparagraph\subparagraph
  \renewcommand{\subparagraph}{
    \@ifstar
      \xxxSubParagraphStar
      \xxxSubParagraphNoStar
  }
  \newcommand{\xxxSubParagraphStar}[1]{\oldsubparagraph*{#1}\mbox{}}
  \newcommand{\xxxSubParagraphNoStar}[1]{\oldsubparagraph{#1}\mbox{}}
\fi
\makeatother

\usepackage{color}
\usepackage{fancyvrb}
\newcommand{\VerbBar}{|}
\newcommand{\VERB}{\Verb[commandchars=\\\{\}]}
\DefineVerbatimEnvironment{Highlighting}{Verbatim}{commandchars=\\\{\}}
% Add ',fontsize=\small' for more characters per line
\newenvironment{Shaded}{}{}
\newcommand{\AlertTok}[1]{\textcolor[rgb]{0.75,0.01,0.01}{\textbf{\colorbox[rgb]{0.97,0.90,0.90}{#1}}}}
\newcommand{\AnnotationTok}[1]{\textcolor[rgb]{0.79,0.38,0.79}{#1}}
\newcommand{\AttributeTok}[1]{\textcolor[rgb]{0.00,0.34,0.68}{#1}}
\newcommand{\BaseNTok}[1]{\textcolor[rgb]{0.69,0.50,0.00}{#1}}
\newcommand{\BuiltInTok}[1]{\textcolor[rgb]{0.39,0.29,0.61}{\textbf{#1}}}
\newcommand{\CharTok}[1]{\textcolor[rgb]{0.57,0.30,0.62}{#1}}
\newcommand{\CommentTok}[1]{\textcolor[rgb]{0.54,0.53,0.53}{#1}}
\newcommand{\CommentVarTok}[1]{\textcolor[rgb]{0.00,0.58,1.00}{#1}}
\newcommand{\ConstantTok}[1]{\textcolor[rgb]{0.67,0.33,0.00}{#1}}
\newcommand{\ControlFlowTok}[1]{\textcolor[rgb]{0.12,0.11,0.11}{\textbf{#1}}}
\newcommand{\DataTypeTok}[1]{\textcolor[rgb]{0.00,0.34,0.68}{#1}}
\newcommand{\DecValTok}[1]{\textcolor[rgb]{0.69,0.50,0.00}{#1}}
\newcommand{\DocumentationTok}[1]{\textcolor[rgb]{0.38,0.47,0.50}{#1}}
\newcommand{\ErrorTok}[1]{\textcolor[rgb]{0.75,0.01,0.01}{\underline{#1}}}
\newcommand{\ExtensionTok}[1]{\textcolor[rgb]{0.00,0.58,1.00}{\textbf{#1}}}
\newcommand{\FloatTok}[1]{\textcolor[rgb]{0.69,0.50,0.00}{#1}}
\newcommand{\FunctionTok}[1]{\textcolor[rgb]{0.39,0.29,0.61}{#1}}
\newcommand{\ImportTok}[1]{\textcolor[rgb]{1.00,0.33,0.00}{#1}}
\newcommand{\InformationTok}[1]{\textcolor[rgb]{0.69,0.50,0.00}{#1}}
\newcommand{\KeywordTok}[1]{\textcolor[rgb]{0.12,0.11,0.11}{\textbf{#1}}}
\newcommand{\NormalTok}[1]{\textcolor[rgb]{0.12,0.11,0.11}{#1}}
\newcommand{\OperatorTok}[1]{\textcolor[rgb]{0.79,0.38,0.79}{#1}}
\newcommand{\OtherTok}[1]{\textcolor[rgb]{0.00,0.43,0.16}{#1}}
\newcommand{\PreprocessorTok}[1]{\textcolor[rgb]{0.00,0.43,0.16}{#1}}
\newcommand{\RegionMarkerTok}[1]{\textcolor[rgb]{0.00,0.34,0.68}{\colorbox[rgb]{0.88,0.91,0.97}{#1}}}
\newcommand{\SpecialCharTok}[1]{\textcolor[rgb]{0.24,0.68,0.91}{#1}}
\newcommand{\SpecialStringTok}[1]{\textcolor[rgb]{1.00,0.33,0.00}{#1}}
\newcommand{\StringTok}[1]{\textcolor[rgb]{0.75,0.01,0.01}{#1}}
\newcommand{\VariableTok}[1]{\textcolor[rgb]{0.00,0.34,0.68}{#1}}
\newcommand{\VerbatimStringTok}[1]{\textcolor[rgb]{0.75,0.01,0.01}{#1}}
\newcommand{\WarningTok}[1]{\textcolor[rgb]{0.75,0.01,0.01}{#1}}

\providecommand{\tightlist}{%
  \setlength{\itemsep}{0pt}\setlength{\parskip}{0pt}}\usepackage{longtable,booktabs,array}
\usepackage{calc} % for calculating minipage widths
% Correct order of tables after \paragraph or \subparagraph
\usepackage{etoolbox}
\makeatletter
\patchcmd\longtable{\par}{\if@noskipsec\mbox{}\fi\par}{}{}
\makeatother
% Allow footnotes in longtable head/foot
\IfFileExists{footnotehyper.sty}{\usepackage{footnotehyper}}{\usepackage{footnote}}
\makesavenoteenv{longtable}
\usepackage{graphicx}
\makeatletter
\newsavebox\pandoc@box
\newcommand*\pandocbounded[1]{% scales image to fit in text height/width
  \sbox\pandoc@box{#1}%
  \Gscale@div\@tempa{\textheight}{\dimexpr\ht\pandoc@box+\dp\pandoc@box\relax}%
  \Gscale@div\@tempb{\linewidth}{\wd\pandoc@box}%
  \ifdim\@tempb\p@<\@tempa\p@\let\@tempa\@tempb\fi% select the smaller of both
  \ifdim\@tempa\p@<\p@\scalebox{\@tempa}{\usebox\pandoc@box}%
  \else\usebox{\pandoc@box}%
  \fi%
}
% Set default figure placement to htbp
\def\fps@figure{htbp}
\makeatother

\usepackage{xspace}
\usepackage{parskip}
\usepackage{doc}
\usepackage{xcolor}
\definecolor{JungleGreen}{RGB}{42,127,98}
\newcommand{\DescribeOption}[4]{
  \begin{minipage}[t]{\textwidth}
    \texttt{#1}
    \textit{\textbf{\textcolor{JungleGreen}{#2}}}\dotfill\,#3\par
    \begingroup
    \vspace{0.5em}#4\par
    \endgroup
  \end{minipage}
}
\KOMAoption{captions}{tableheading}
\makeatletter
\@ifpackageloaded{bookmark}{}{\usepackage{bookmark}}
\makeatother
\makeatletter
\@ifpackageloaded{caption}{}{\usepackage{caption}}
\AtBeginDocument{%
\ifdefined\contentsname
  \renewcommand*\contentsname{Table of contents}
\else
  \newcommand\contentsname{Table of contents}
\fi
\ifdefined\listfigurename
  \renewcommand*\listfigurename{List of Figures}
\else
  \newcommand\listfigurename{List of Figures}
\fi
\ifdefined\listtablename
  \renewcommand*\listtablename{List of Tables}
\else
  \newcommand\listtablename{List of Tables}
\fi
\ifdefined\figurename
  \renewcommand*\figurename{Figure}
\else
  \newcommand\figurename{Figure}
\fi
\ifdefined\tablename
  \renewcommand*\tablename{Table}
\else
  \newcommand\tablename{Table}
\fi
}
\@ifpackageloaded{float}{}{\usepackage{float}}
\floatstyle{ruled}
\@ifundefined{c@chapter}{\newfloat{codelisting}{h}{lop}}{\newfloat{codelisting}{h}{lop}[chapter]}
\floatname{codelisting}{Listing}
\newcommand*\listoflistings{\listof{codelisting}{List of Listings}}
\makeatother
\makeatletter
\makeatother
\makeatletter
\@ifpackageloaded{caption}{}{\usepackage{caption}}
\@ifpackageloaded{subcaption}{}{\usepackage{subcaption}}
\makeatother

\usepackage{bookmark}

\IfFileExists{xurl.sty}{\usepackage{xurl}}{} % add URL line breaks if available
\urlstyle{same} % disable monospaced font for URLs
\hypersetup{
  pdftitle={Moloch},
  pdfauthor={Johan Larsson; Matthias Vogelgesang},
  colorlinks=true,
  linkcolor={blue},
  filecolor={Maroon},
  citecolor={Blue},
  urlcolor={Blue},
  pdfcreator={LaTeX via pandoc}}


\title{Moloch}
\usepackage{etoolbox}
\makeatletter
\providecommand{\subtitle}[1]{% add subtitle to \maketitle
  \apptocmd{\@title}{\par {\large #1 \par}}{}{}
}
\makeatother
\subtitle{A clean and simple Beamer theme}
\author{Johan Larsson \and Matthias Vogelgesang}
\date{2 December 2025}

\begin{document}
\maketitle

\renewcommand*\contentsname{Table of contents}
{
\hypersetup{linkcolor=}
\setcounter{tocdepth}{2}
\tableofcontents
}

\bookmarksetup{startatroot}

\chapter{Introduction}\label{introduction}

Beamer is a great way to make presentations with LaTeX, but its theme
selection is surprisingly sparse. The stock themes share an aesthetic
that can be a little cluttered, while the few distinctive custom themes
available are often specialized for a particular corporate or
institutional brand.

The goal of Moloch is to provide a simple, modern Beamer theme suitable
for anyone to use. It tries to minimize noise and maximize space for
content; the only visual flourish it offers is an (optional) progress
bar added to each slide or to the section slides.

Moloch's codebase is maintained at
\url{https://github.com/jolars/moloch}. If you have any issues, find
mistakes in the manual or want to help make the theme even better,
please get in touch there.

Moloch is a fork of the popular Metropolis theme by Matthias
Vogelgesang. The motivation for the fork was to fix some longstanding
bugs in Metropolis and also simplify the codebase to make it easier to
maintain and less fragile to changes in the underlying Beamer code.

\bookmarksetup{startatroot}

\chapter{Getting Started}\label{getting-started}

\section{Installing from CTAN}\label{installing-from-ctan}

For most users, we recommend installing Moloch from
\href{https://www.ctan.org}{CTAN}. If you keep your TeX~distribution
up-to-date, chances are good that Moloch is already installed. If it is
not, you need to update your packages. If your distribution is TeX~Live
(or MacTeX~on OS X), the following command updates all packages.

\begin{Shaded}
\begin{Highlighting}[]
\ExtensionTok{tlmgr}\NormalTok{ update }\AttributeTok{{-}{-}all}
\end{Highlighting}
\end{Shaded}

If this results in an error, you may need to run it with administrative
privileges:

\begin{Shaded}
\begin{Highlighting}[]
\FunctionTok{sudo}\NormalTok{ tlmgr update }\AttributeTok{{-}{-}all}
\end{Highlighting}
\end{Shaded}

MacTeX~on OS X also provides a graphical interface for \texttt{tlmgr}
called TeX~Live Utility.

For any other distribution please refer to its documentation on how to
update your packages.

\section{Installing from Source}\label{installing-from-source}

If you want to use the development version of Moloch, you can install it
manually. You only need a recent LaTeX~distribution which includes
\textbf{l3build}. Then simply follow the steps below.

Download the source with a \texttt{git\ clone} of
\url{https://github.com/jolars/moloch}

Install the package by running

\begin{Shaded}
\begin{Highlighting}[]
\ExtensionTok{l3build}\NormalTok{ install}
\end{Highlighting}
\end{Shaded}

inside the downloaded directory.

\section{A Minimal Example}\label{a-minimal-example}

The following code shows a minimal example of a Beamer presentation
using Moloch.

\begin{Shaded}
\begin{Highlighting}[]
\BuiltInTok{\textbackslash{}documentclass}\NormalTok{\{}\ExtensionTok{beamer}\NormalTok{\}}
\FunctionTok{\textbackslash{}usetheme}\NormalTok{\{moloch\}}

\FunctionTok{\textbackslash{}title}\NormalTok{\{A Minimal Example\}}

\FunctionTok{\textbackslash{}date}\NormalTok{\{}\FunctionTok{\textbackslash{}today}\NormalTok{\}}
\FunctionTok{\textbackslash{}author}\NormalTok{\{Johan Larsson\}}
\FunctionTok{\textbackslash{}institute}\NormalTok{\{Some University\}}

\KeywordTok{\textbackslash{}begin}\NormalTok{\{}\ExtensionTok{document}\NormalTok{\}}
    \FunctionTok{\textbackslash{}maketitle}

    \KeywordTok{\textbackslash{}section}\NormalTok{\{First Section\}}

    \KeywordTok{\textbackslash{}begin}\NormalTok{\{}\ExtensionTok{frame}\NormalTok{\}}
        \FunctionTok{\textbackslash{}frametitle}\NormalTok{\{First Frame\}}

\NormalTok{        Hello, world!}

    \KeywordTok{\textbackslash{}end}\NormalTok{\{}\ExtensionTok{frame}\NormalTok{\}}
\KeywordTok{\textbackslash{}end}\NormalTok{\{}\ExtensionTok{document}\NormalTok{\}}
\end{Highlighting}
\end{Shaded}

\section{Dependencies}\label{dependencies}

Moloch depends on the \texttt{beamer} class and the following standard
packages:

\begin{itemize}
\tightlist
\item
  \href{https://ctan.org/pkg/pgf}{tikz}
\item
  \href{https://ctan.org/pkg/pgfopts}{pgfopts}
\item
  \href{https://ctan.org/pkg/etoolbox}{etoolbox}
\item
  \href{https://ctan.org/pkg/calc}{calc}
\end{itemize}

\section{Pandoc}\label{pandoc}

To use this theme with
\href{http://johnmacfarlane.net/pandoc/}{Pandoc}-based presentations,
you can run the following command in your terminal:

\begin{Shaded}
\begin{Highlighting}[]
\ExtensionTok{pandoc} \AttributeTok{{-}t}\NormalTok{ beamer }\AttributeTok{{-}V}\NormalTok{ theme:moloch }\AttributeTok{{-}o}\NormalTok{ output.pdf input.md}
\end{Highlighting}
\end{Shaded}

\bookmarksetup{startatroot}

\chapter{Customization}\label{customization}

\section{Package Options}\label{package-options}

The theme provides a number of options, which can be set using a
key=value interface. The primary way to set options is to provide a
comma-separated list of option-value pairs when loading Moloch in the
preamble:

\begin{Shaded}
\begin{Highlighting}[]
\FunctionTok{\textbackslash{}usetheme}\NormalTok{[}
\NormalTok{    option1=value1,}
\NormalTok{    option2=value2,}
\NormalTok{    ...}
\NormalTok{]\{moloch\}}
\end{Highlighting}
\end{Shaded}

Options can be changed at any time---even mid-presentation---with the
\texttt{molochset()} macro.

\begin{Shaded}
\begin{Highlighting}[]
\FunctionTok{\textbackslash{}molochset}\NormalTok{\{}
\NormalTok{    option1=newvalue1,}
\NormalTok{    option2=newvalue2,}
\NormalTok{    ...}
\NormalTok{\}}
\end{Highlighting}
\end{Shaded}

The list of options is structured as shown in the following example.

\DescribeOption{<option-name>}{<list>, <of>, <values>}{<default>}{A short description of the option.}

\subsection{Main Theme}\label{main-theme}

\DescribeOption{titleformat}{regular, smallcaps, allsmallcaps, allcaps}{regular}{Changes the format of titles, subtitles, section titles, frame titles,
and the text on ``standout'' frames. The available options produce
Regular, \textsc{SmallCaps}, \textsc{allsmallcaps}, or ALLCAPS titles.
Note that these commands do not affect math and numbers, so may not work
as you expect if your titles contain these.}

\DescribeOption{standoutnumberformat}{regular, smallcaps, allsmallcaps, allcaps}{regular}{Changes the format of ``standout'' frames (see \texttt{titleformat},
above).}

\subsection{Inner Theme}\label{inner-theme}

\DescribeOption{sectionpage}{none, simple, progressbar}{progressbar}{Adds a slide at the start of each section (\texttt{simple}) with an
optional thin progress bar below the section title
(\texttt{progressbar}). The \texttt{none} option disables the section
page.}

\DescribeOption{subsectionpage}{none, simple, progressbar}{none}{Optionally adds a slide at the start of each subsection. If enabled with
the \texttt{simple} or \texttt{progressbar} options, the style of the
\texttt{section\ page} will be updated to match the style of the
\texttt{subsection\ page}. Note that section slides and subsection
slides can appear consecutively if both are enabled; you may want to use
this option together with \texttt{sectionpage=none} depending on the
section structure of your presentation.}

\DescribeOption{standoutnumbering}{none, hide, show}{none}{This option decides whether or not to count standout pages as frames if
frame counting. Option \texttt{none} (the default) means that the
standout frames are not counted. \texttt{hide} means that they are
counted but that there won't be any footer showing a frame number.
\texttt{show} means that they are counted and that the frame number
count is shown in the same fashion as for regular frames.}

\subsection{Outer Theme}\label{outer-theme}

\DescribeOption{numbering}{none, counter, fraction}{(none specified)}{\emph{This option is deprecated and will be removed in a future version.
Please use Beamer's \texttt{page\ number\ in\ head/foot} template
instead.} Controls whether the frame number at the bottom right of each
slide is omitted (\texttt{none}), shown (\texttt{counter}) or displayed
as a fraction of the total number of frames (\texttt{fraction}).}

\DescribeOption{progressbar}{none, head, frametitle, foot}{none}{Optionally adds a progress bar to the top of each frame (\texttt{head}),
the bottom of each frame (\texttt{foot}), or directly below each frame
title (\texttt{frametitle}).}

\subsection{Color Theme}\label{color-theme}

\DescribeOption{block}{transparent, fill}{transparent}{Optionally adds a light grey background to block environments like
\texttt{theorem} and \texttt{example}.}

\DescribeOption{background}{dark, light}{light}{Provides the option to have a dark background and light foreground
instead of the reverse.}

\subsection{Font Theme}\label{font-theme}

\DescribeOption{titleformat plain, titleformat frametitle, titleformat section}{regular, smallcaps, allsmallcaps, allcaps}{regular}{Individually controls the format of titles, subtitles, section titles,
and frame titles (see \texttt{titleformat}, above).}

\section{Color Customization}\label{color-customization}

The included Moloch color theme is used by default, but its colors can
be easily changed to suit your tastes. All of the theme's styles are
defined in terms of three beamer colors:

\begin{itemize}
\tightlist
\item
  \texttt{normal\ text} (dark fg, light bg)
\item
  \texttt{alerted\ text} (colored fg, should be visible against dark or
  light)
\item
  \texttt{example\ text} (colored fg, should be visible against dark or
  light)
\end{itemize}

An easy way to customize the theme is to redefine these colors using

\begin{Shaded}
\begin{Highlighting}[]
\FunctionTok{\textbackslash{}setbeamercolor}\NormalTok{\{ ... \}\{ fg= ... , bg= ... \}}
\end{Highlighting}
\end{Shaded}

in your preamble. For greater customization, you can redefine any of the
other stock beamer colors. In addition to the stock colors the theme
defines a number of Moloch specific colors, which can also be redefined
to your liking.

\begin{Shaded}
\begin{Highlighting}[]
\FunctionTok{\textbackslash{}setbeamercolor}\NormalTok{\{progress bar\}\{ ... \}}
\FunctionTok{\textbackslash{}setbeamercolor}\NormalTok{\{title separator\}\{ ... \}}
\FunctionTok{\textbackslash{}setbeamercolor}\NormalTok{\{progress bar in head/foot\}\{ ... \}}
\FunctionTok{\textbackslash{}setbeamercolor}\NormalTok{\{progress bar in section page\}\{ ... \}}
\end{Highlighting}
\end{Shaded}

\subsection{Themes}\label{themes}

For low-light situations Moloch it might be helpful to use the
\texttt{moloch-highcontrast} color theme. It is enabled like any other
color theme:

\begin{Shaded}
\begin{Highlighting}[]
\FunctionTok{\textbackslash{}usecolortheme}\NormalTok{\{moloch{-}highcontrast\}}
\end{Highlighting}
\end{Shaded}

There is also a theme based on the
\href{tomorrow\%20color\%20theme}{https://github.com/chriskempson/tomorrow-theme},
which you can enable like this:

\begin{Shaded}
\begin{Highlighting}[]
\FunctionTok{\textbackslash{}usecolortheme}\NormalTok{\{moloch{-}tomorrow\}}
\end{Highlighting}
\end{Shaded}

\section{Commands}\label{commands}

\subsection{Standout Frames}\label{standout-frames}

The Moloch inner theme offers a custom frame format with large, centered
text and an inverted background---perfect for focusing attention on
single sentence or image. To use it, add the key \texttt{standout} to
the frame:

\begin{Shaded}
\begin{Highlighting}[]
\KeywordTok{\textbackslash{}begin}\NormalTok{\{}\ExtensionTok{frame}\NormalTok{\}[standout]}
\NormalTok{    Thank you!}
\KeywordTok{\textbackslash{}end}\NormalTok{\{}\ExtensionTok{frame}\NormalTok{\}}
\end{Highlighting}
\end{Shaded}

\bookmarksetup{startatroot}

\chapter{Known Issues}\label{known-issues}

\section{Title Formats}\label{title-formats}

Be aware that not every font supports small caps, so the
\texttt{smallcaps} or \texttt{allsmallcaps} options may not work for all
fonts. In particular, the Computer Modern sans-serif typeface, which is
used by default when Moloch is compiled with pdfLaTeX, does not have a
small-caps variant.

Note that title format options \texttt{allsmallcaps} and
\texttt{allcaps} do not affect the sizes of numerals, punctuation, and
math symbol, and are probably best avoided if your titles contain these
characters.

\section{Interactions with Other Color
Themes}\label{interactions-with-other-color-themes}

Moloch can be used along with any other Beamer color theme, such as
\texttt{crane} or \texttt{seahorse}. If you wish to do this, it is
usually best to include the Moloch subpackages individually so the
Moloch color theme is never loaded. This will prevent conflicts between
the Moloch color theme and your preferred theme.

For example, overriding the color theme as follows may not work as
expected because
`\texttt{loads\ the\ Moloch\ color\ theme,\ which\ defines\ a\ relationship\ between\ the\ frametitle\ background\ and\ the\ primary\ palette\ of\ the\ theme.\ Since}seahorse`
assumes a different relationship between its palettes, the result is a
grey, rather than periwinkle, frametitle background.

\begin{Shaded}
\begin{Highlighting}[]
\FunctionTok{\textbackslash{}usetheme}\NormalTok{\{moloch\}}
\FunctionTok{\textbackslash{}usecolortheme}\NormalTok{\{seahorse\}}
\end{Highlighting}
\end{Shaded}

The correct colors are chosen if the Moloch outer, inner, and font
themes are loaded seperately:

\begin{Shaded}
\begin{Highlighting}[]
\FunctionTok{\textbackslash{}useoutertheme}\NormalTok{\{moloch\}}
\FunctionTok{\textbackslash{}useinnertheme}\NormalTok{\{moloch\}}
\FunctionTok{\textbackslash{}usefonttheme}\NormalTok{\{moloch\}}
\FunctionTok{\textbackslash{}usecolortheme}\NormalTok{\{seahorse\}   }\CommentTok{\% or your preferred color theme}
\end{Highlighting}
\end{Shaded}

Please note that Moloch may not use all the colors defined in your
favourite Beamer color theme. In particular, Moloch does not set a
background color for the title; this will cause issues when using color
themes like \texttt{whale} which set a white foreground for the title.

\section{Notes on Second Screen}\label{notes-on-second-screen}

If you use the \texttt{{[}show\ notes\ on\ second\ screen{]}} option
built in to Beamer and compile with XeLaTeX, text on slides following
the first section slide may be rendered in white instead of the regular
colour. This is due to
\href{http://tex.stackexchange.com/questions/288408/}{a bug} in Beamer
or ~itself. You can work around it either by compiling with ~or by
adding the following code to your preamble to reset the text color on
each slide.

\begin{Shaded}
\begin{Highlighting}[]
\FunctionTok{\textbackslash{}makeatletter}
\FunctionTok{\textbackslash{}def\textbackslash{}beamer@framenotesbegin}\NormalTok{\{}\CommentTok{\% at beginning of slide}
  \FunctionTok{\textbackslash{}usebeamercolor}\NormalTok{[fg]\{normal text\}}
  \FunctionTok{\textbackslash{}gdef\textbackslash{}beamer@noteitems}\NormalTok{\{\}}\CommentTok{\%}
  \FunctionTok{\textbackslash{}gdef\textbackslash{}beamer@notes}\NormalTok{\{\}}\CommentTok{\%}
\NormalTok{\}}
\FunctionTok{\textbackslash{}makeatother}
\end{Highlighting}
\end{Shaded}

\section{Standout Frames with Labels}\label{standout-frames-with-labels}

Because the \texttt{standout} frame option creates a group to restrict
the colour change to a single slide, labels defined after calling
\texttt{standout} will stay local to the group. In other words, the
following may result in a ``label undefined'' error.

\begin{Shaded}
\begin{Highlighting}[]
\KeywordTok{\textbackslash{}begin}\NormalTok{\{}\ExtensionTok{frame}\NormalTok{\}[standout, label=conclusion]\{Conclusion\}}
\NormalTok{  Awesome slide}
\KeywordTok{\textbackslash{}end}\NormalTok{\{}\ExtensionTok{frame}\NormalTok{\}}
\end{Highlighting}
\end{Shaded}

To fix this problem, change the order of the keys in the frame.

\begin{Shaded}
\begin{Highlighting}[]
\KeywordTok{\textbackslash{}begin}\NormalTok{\{}\ExtensionTok{frame}\NormalTok{\}[label=conclusion, standout]\{Conclusion\}}
\NormalTok{  Awesome slide}
\KeywordTok{\textbackslash{}end}\NormalTok{\{}\ExtensionTok{frame}\NormalTok{\}}
\end{Highlighting}
\end{Shaded}

This error can be unwittingly triggered if you export your slides from
Emacs Org mode, which automatically adds labels after frame options.
Alex Branham \href{https://github.com/matze/mtheme/issues/203}{offers}
the following solution for Org mode users, using
\texttt{org-set-property}.

\begin{Shaded}
\begin{Highlighting}[]
\FunctionTok{* Start of a frame}
\NormalTok{    :PROPERTIES:}
\NormalTok{    :BEAMER\_opt: label=conclusion,standout}
\NormalTok{    :END:}
\end{Highlighting}
\end{Shaded}

\section{Standout Frames with Pandoc}\label{standout-frames-with-pandoc}

With Pandoc versions prior to 1.17.2 it was not possible to create
standout frames because Pandoc only supported a specific list of frame
attributes thus ignoring additional attributes such as
\texttt{.standout}.

\section{License}\label{license}

Moloch is licensed under a
\href{http://creativecommons.org/licenses/by-sa/4.0/}{Creative Commons
Attribution-ShareAlike 4.0 International License}. This means that if
you change the theme and re-distribute it, you must retain the copyright
notice header and license it under the same CC-BY-SA license. This does
not affect any presentations that you create with the theme.

\section{Implementation}\label{implementation}

\cleardoublepage
\phantomsection
\addcontentsline{toc}{part}{Appendices}
\appendix

\part{Implementation}

\chapter{Main Theme}\label{main-theme-1}

The primary job of this package is to load the component sub-packages of
the theme and route the theme options accordingly. It also provides some
custom commands and environments for the user.

\subsection{Package Dependencies}\label{package-dependencies}

\begin{Shaded}
\begin{Highlighting}[]
\FunctionTok{\textbackslash{}RequirePackage}\NormalTok{\{pgfopts\}}
\end{Highlighting}
\end{Shaded}

\subsection{Options}\label{options}

Most options are passed off to the component sub-packages.

\begin{Shaded}
\begin{Highlighting}[]
\FunctionTok{\textbackslash{}pgfkeys}\NormalTok{\{/moloch/.cd,}
\NormalTok{  .search also=\{}
\NormalTok{      /moloch/inner,}
\NormalTok{      /moloch/outer,}
\NormalTok{      /moloch/color,}
\NormalTok{      /moloch/font,}
\NormalTok{    \}}
\NormalTok{\}}
\end{Highlighting}
\end{Shaded}

\begin{description}
\tightlist
\item[\texttt{titleformat\ plain}]
Controls the formatting of the text on standout ``plain'' frames.
\end{description}

\begin{Shaded}
\begin{Highlighting}[]
\FunctionTok{\textbackslash{}pgfkeys}\NormalTok{\{}
\NormalTok{  /moloch/titleformat plain/.cd,}
\NormalTok{  .is choice,}
\NormalTok{  regular/.code=\{}\CommentTok{\%}
      \FunctionTok{\textbackslash{}let\textbackslash{}moloch@plaintitleformat\textbackslash{}@empty}\CommentTok{\%}
      \FunctionTok{\textbackslash{}setbeamerfont}\NormalTok{\{standout\}\{shape=}\FunctionTok{\textbackslash{}normalfont}\NormalTok{\}}\CommentTok{\%}
\NormalTok{    \},}
\NormalTok{  smallcaps/.code=\{}\CommentTok{\%}
      \FunctionTok{\textbackslash{}let\textbackslash{}moloch@plaintitleformat\textbackslash{}@empty}\CommentTok{\%}
      \FunctionTok{\textbackslash{}setbeamerfont}\NormalTok{\{standout\}\{shape=}\FunctionTok{\textbackslash{}scshape}\NormalTok{\}}\CommentTok{\%}
\NormalTok{    \},}
\NormalTok{  allsmallcaps/.code=\{}\CommentTok{\%}
      \FunctionTok{\textbackslash{}let\textbackslash{}moloch@plaintitleformat\textbackslash{}MakeLowercase}\CommentTok{\%}
      \FunctionTok{\textbackslash{}setbeamerfont}\NormalTok{\{standout\}\{shape=}\FunctionTok{\textbackslash{}scshape}\NormalTok{\}}\CommentTok{\%}
\NormalTok{    \},}
\NormalTok{  allcaps/.code=\{}\CommentTok{\%}
      \FunctionTok{\textbackslash{}let\textbackslash{}moloch@plaintitleformat\textbackslash{}MakeUppercase}\CommentTok{\%}
      \FunctionTok{\textbackslash{}setbeamerfont}\NormalTok{\{standout\}\{shape=}\FunctionTok{\textbackslash{}normalfont}\NormalTok{\}}\CommentTok{\%}
\NormalTok{    \},}
\NormalTok{\}}
\end{Highlighting}
\end{Shaded}

\begin{description}
\tightlist
\item[\texttt{titleformat}]
Sets a standard format for titles, subtitles, section titles, frame
titles, and the text on standout ``plain'' frames.
\end{description}

\begin{Shaded}
\begin{Highlighting}[]
\FunctionTok{\textbackslash{}pgfkeys}\NormalTok{\{}
\NormalTok{  /moloch/titleformat/.code=}\FunctionTok{\textbackslash{}pgfkeysalso}\NormalTok{\{}
\NormalTok{    font/titleformat title=\#1,}
\NormalTok{    font/titleformat subtitle=\#1,}
\NormalTok{    font/titleformat section=\#1,}
\NormalTok{    font/titleformat frame=\#1,}
\NormalTok{    titleformat plain=\#1,}
\NormalTok{  \}}
\NormalTok{\}}
\end{Highlighting}
\end{Shaded}

Set default values for options.

\begin{Shaded}
\begin{Highlighting}[]
\FunctionTok{\textbackslash{}newcommand}\NormalTok{\{}\ExtensionTok{\textbackslash{}moloch@setdefaults}\NormalTok{\}\{}
  \FunctionTok{\textbackslash{}pgfkeys}\NormalTok{\{/moloch/.cd,}
\NormalTok{    titleformat plain=regular,}
\NormalTok{  \}}
\NormalTok{\}}
\end{Highlighting}
\end{Shaded}

\subsection{Component Sub-Packages}\label{component-sub-packages}

Having processed the options, we can now load the component sub-packages
of the theme.

\begin{Shaded}
\begin{Highlighting}[]
\FunctionTok{\textbackslash{}useinnertheme}\NormalTok{\{moloch\}}
\FunctionTok{\textbackslash{}useoutertheme}\NormalTok{\{moloch\}}
\FunctionTok{\textbackslash{}usecolortheme}\NormalTok{\{moloch\}}
\FunctionTok{\textbackslash{}usefonttheme}\NormalTok{\{moloch\}}
\end{Highlighting}
\end{Shaded}

\subsection{Custom Commands}\label{custom-commands}

The parent theme defines custom commands as their proper usage may
depend on multiple sub-packages.

\begin{description}
\tightlist
\item[\texttt{Allows}]
the user to change options midway through a presentation.
\end{description}

\begin{Shaded}
\begin{Highlighting}[]
\FunctionTok{\textbackslash{}newcommand}\NormalTok{\{}\ExtensionTok{\textbackslash{}molochset}\NormalTok{\}[1]\{}\FunctionTok{\textbackslash{}pgfkeys}\NormalTok{\{/moloch/.cd,\#1\}\}}
\end{Highlighting}
\end{Shaded}

\begin{Shaded}
\begin{Highlighting}[]
\FunctionTok{\textbackslash{}newcommand}\NormalTok{\{}\ExtensionTok{\textbackslash{}mreducelistspacing}\NormalTok{\}\{}\FunctionTok{\textbackslash{}vspace}\NormalTok{\{{-}}\FunctionTok{\textbackslash{}topsep}\NormalTok{\}\}}
\end{Highlighting}
\end{Shaded}

\subsection{Process Package Options}\label{process-package-options}

\begin{Shaded}
\begin{Highlighting}[]
\FunctionTok{\textbackslash{}moloch@setdefaults}
\FunctionTok{\textbackslash{}ProcessPgfOptions}\NormalTok{\{/moloch\}}
\end{Highlighting}
\end{Shaded}

\chapter{Inner Theme}\label{inner-theme-1}

A \texttt{beamer} inner theme dictates the style of the frame elements
traditionally set in the ``body'' of each slide. These include:

\begin{itemize}
\item
  title, part, and section pages;
\item
  itemize, enumerate, and description environments;
\item
  block environments including theorems and proofs;
\item
  figures and tables; and
\item
  footnotes and plain text.
\end{itemize}

\subsection{Package Dependencies}\label{package-dependencies-1}

\begin{Shaded}
\begin{Highlighting}[]
\FunctionTok{\textbackslash{}RequirePackage}\NormalTok{\{keyval\}}
\FunctionTok{\textbackslash{}RequirePackage}\NormalTok{\{calc\}}
\FunctionTok{\textbackslash{}RequirePackage}\NormalTok{\{pgfopts\}}
\FunctionTok{\textbackslash{}RequirePackage}\NormalTok{\{tikz\}}
\end{Highlighting}
\end{Shaded}

\subsection{Memoization and Tikz
Externalization}\label{memoization-and-tikz-externalization}

See the documentation for the correspondign section under the outer
theme for more information on the following lines.

\begin{Shaded}
\begin{Highlighting}[]
\FunctionTok{\textbackslash{}providecommand}\NormalTok{\{}\ExtensionTok{\textbackslash{}tikzexternalenable}\NormalTok{\}\{\}}
\FunctionTok{\textbackslash{}providecommand}\NormalTok{\{}\ExtensionTok{\textbackslash{}tikzexternaldisable}\NormalTok{\}\{\}}
\FunctionTok{\textbackslash{}providecommand}\NormalTok{\{}\ExtensionTok{\textbackslash{}mmzUnmemoizable}\NormalTok{\}\{\}}
\end{Highlighting}
\end{Shaded}

\subsection{Options}\label{options-1}

\begin{description}
\tightlist
\item[\texttt{sectionpage}]
Optionally add a slide marking the beginning of each section.
\end{description}

\begin{Shaded}
\begin{Highlighting}[]
\FunctionTok{\textbackslash{}pgfkeys}\NormalTok{\{}
\NormalTok{  /moloch/inner/sectionpage/.cd,}
\NormalTok{  .is choice,}
\NormalTok{  none/.code=}\FunctionTok{\textbackslash{}moloch@disablesectionpage}\NormalTok{,}
\NormalTok{  simple/.code=\{}\CommentTok{\%}
      \FunctionTok{\textbackslash{}moloch@enablesectionpage}\CommentTok{\%}
      \FunctionTok{\textbackslash{}setbeamertemplate}\NormalTok{\{section page\}[simple]}\CommentTok{\%}
\NormalTok{    \},}
\NormalTok{  progressbar/.code=\{}\CommentTok{\%}
      \FunctionTok{\textbackslash{}moloch@enablesectionpage}\CommentTok{\%}
      \FunctionTok{\textbackslash{}setbeamertemplate}\NormalTok{\{section page\}[progressbar]}\CommentTok{\%}
\NormalTok{    \},}
\NormalTok{\}}
\end{Highlighting}
\end{Shaded}

\begin{description}
\tightlist
\item[\texttt{subsectionpage}]
Optionally add a slide marking the beginning of each subsection.
\end{description}

\begin{Shaded}
\begin{Highlighting}[]
\FunctionTok{\textbackslash{}pgfkeys}\NormalTok{\{}
\NormalTok{  /moloch/inner/subsectionpage/.cd,}
\NormalTok{  .is choice,}
\NormalTok{  none/.code=}\FunctionTok{\textbackslash{}moloch@disablesubsectionpage}\NormalTok{,}
\NormalTok{  simple/.code=\{}\CommentTok{\%}
      \FunctionTok{\textbackslash{}moloch@enablesubsectionpage}\CommentTok{\%}
      \FunctionTok{\textbackslash{}setbeamertemplate}\NormalTok{\{section page\}[simple]}\CommentTok{\%}
\NormalTok{    \},}
\NormalTok{  progressbar/.code=\{}\CommentTok{\%}
      \FunctionTok{\textbackslash{}moloch@enablesubsectionpage}\CommentTok{\%}
      \FunctionTok{\textbackslash{}setbeamertemplate}\NormalTok{\{section page\}[progressbar]}\CommentTok{\%}
\NormalTok{    \},}
\NormalTok{\}}
\end{Highlighting}
\end{Shaded}

\begin{description}
\tightlist
\item[\texttt{standoutnumbering}]
Whether or not to number standout pages. Option \texttt{none} means that
standout pages are not numbered (do not count as frames). \texttt{hide}
means that they do count as frames, but that the footer with the number
is not shown. Option \texttt{show} means that they both count as frames
and that the footer with a frame count is shown.
\end{description}

\begin{Shaded}
\begin{Highlighting}[]
\FunctionTok{\textbackslash{}providebool}\NormalTok{\{moloch@enableStandoutFooter\}}
\FunctionTok{\textbackslash{}providebool}\NormalTok{\{moloch@enableStandoutNumbering\}}
\FunctionTok{\textbackslash{}pgfkeys}\NormalTok{\{}
\NormalTok{  /moloch/inner/standoutnumbering/.cd,}
\NormalTok{  .is choice,}
\NormalTok{  none/.code=\{}
      \FunctionTok{\textbackslash{}boolfalse}\NormalTok{\{moloch@enableStandoutNumbering\}}
      \FunctionTok{\textbackslash{}boolfalse}\NormalTok{\{moloch@enableStandoutFooter\}}
\NormalTok{    \},}
\NormalTok{  show/.code=\{}
      \FunctionTok{\textbackslash{}booltrue}\NormalTok{\{moloch@enableStandoutNumbering\}}
      \FunctionTok{\textbackslash{}booltrue}\NormalTok{\{moloch@enableStandoutFooter\}}
\NormalTok{    \},}
\NormalTok{  hide/.code=\{}
      \FunctionTok{\textbackslash{}booltrue}\NormalTok{\{moloch@enableStandoutNumbering\}}
      \FunctionTok{\textbackslash{}boolfalse}\NormalTok{\{moloch@enableStandoutFooter\}}
\NormalTok{    \}}
\NormalTok{\}}
\end{Highlighting}
\end{Shaded}

\begin{description}
\tightlist
\item[\texttt{titleseparator\ linewidth}]
Set the width of the line separating the title from the author.
\end{description}

\begin{Shaded}
\begin{Highlighting}[]
\FunctionTok{\textbackslash{}newlength}\NormalTok{\{}\FunctionTok{\textbackslash{}moloch@titleseparator@linewidth}\NormalTok{\}}
\FunctionTok{\textbackslash{}pgfkeys}\NormalTok{\{}
\NormalTok{  /moloch/inner/.cd,}
\NormalTok{  titleseparatorlinewidth/.code=\{}\FunctionTok{\textbackslash{}setlength}\NormalTok{\{}\FunctionTok{\textbackslash{}moloch@titleseparator@linewidth}\NormalTok{\}\{\#1\}\},}
\NormalTok{  titleseparatorlinewidth/.default=0.4pt,}
\NormalTok{\}}
\end{Highlighting}
\end{Shaded}

\begin{description}
\tightlist
\item[\texttt{titleseparator\ aliases}]
Allows \texttt{titleseparator\ linewidth} to be used in
\texttt{\textbackslash{}usetheme}.
\end{description}

\begin{Shaded}
\begin{Highlighting}[]
\FunctionTok{\textbackslash{}pgfkeys}\NormalTok{\{}
\NormalTok{  /moloch/inner/.cd,}
\NormalTok{  titleseparator linewidth/.code=}\FunctionTok{\textbackslash{}pgfkeysalso}\NormalTok{\{titleseparatorlinewidth=\#1\},}
\NormalTok{\}}
\end{Highlighting}
\end{Shaded}

\begin{description}
\tightlist
\item[\texttt{@inner@setdefaults}]
Set default values for inner theme options.
\end{description}

\begin{Shaded}
\begin{Highlighting}[]
\FunctionTok{\textbackslash{}newcommand}\NormalTok{\{}\ExtensionTok{\textbackslash{}moloch@inner@setdefaults}\NormalTok{\}\{}
  \FunctionTok{\textbackslash{}pgfkeys}\NormalTok{\{/moloch/inner/.cd,}
\NormalTok{    sectionpage=progressbar,}
\NormalTok{    subsectionpage=none,}
\NormalTok{    standoutnumbering=none,}
\NormalTok{    titleseparator linewidth=0.4pt,}
\NormalTok{  \}}
\NormalTok{\}}
\end{Highlighting}
\end{Shaded}

\subsection{Title Page}\label{title-page}

\begin{description}
\tightlist
\item[\texttt{title\ page}]
Template for the title page. Each element is only typset if it is
defined by the user. If \texttt{\textbackslash{}subtitle} is empty, for
example, it won't leave a blank space on the title slide.
\end{description}

\begin{Shaded}
\begin{Highlighting}[]
\FunctionTok{\textbackslash{}setbeamertemplate}\NormalTok{\{title page\}\{}
  \FunctionTok{\textbackslash{}null}\CommentTok{\%}
  \FunctionTok{\textbackslash{}vspace}\NormalTok{\{0pt plus 1.618fil\}}\CommentTok{\%}
  \FunctionTok{\textbackslash{}vfil}\CommentTok{\%}
  \FunctionTok{\textbackslash{}ifx\textbackslash{}inserttitlegraphic\textbackslash{}@empty\textbackslash{}else\textbackslash{}usebeamertemplate*}\NormalTok{\{title graphic\}}\FunctionTok{\textbackslash{}fi}
  \FunctionTok{\textbackslash{}ifx\textbackslash{}inserttitle\textbackslash{}@empty\textbackslash{}else\textbackslash{}usebeamertemplate*}\NormalTok{\{title\}}\FunctionTok{\textbackslash{}fi}
  \FunctionTok{\textbackslash{}ifx\textbackslash{}insertsubtitle\textbackslash{}@empty\textbackslash{}else\textbackslash{}usebeamertemplate*}\NormalTok{\{subtitle\}}\FunctionTok{\textbackslash{}fi}
  \FunctionTok{\textbackslash{}usebeamertemplate*}\NormalTok{\{title separator\}}
  \FunctionTok{\textbackslash{}expandafter\textbackslash{}ifblank\textbackslash{}expandafter}\NormalTok{\{}\FunctionTok{\textbackslash{}beamer@andstripped}\NormalTok{\}\{\}\{}\CommentTok{\%}
    \FunctionTok{\textbackslash{}usebeamertemplate*}\NormalTok{\{author\}}\CommentTok{\%}
\NormalTok{  \}}
  \FunctionTok{\textbackslash{}ifx\textbackslash{}insertinstitute\textbackslash{}@empty\textbackslash{}else\textbackslash{}usebeamertemplate*}\NormalTok{\{institute\}}\FunctionTok{\textbackslash{}fi}
  \FunctionTok{\textbackslash{}ifx\textbackslash{}insertdate\textbackslash{}@empty\textbackslash{}else\textbackslash{}usebeamertemplate*}\NormalTok{\{date\}}\FunctionTok{\textbackslash{}fi}
  \FunctionTok{\textbackslash{}vspace}\NormalTok{\{0pt plus 1fil\}}\CommentTok{\%}
  \FunctionTok{\textbackslash{}null}
\NormalTok{\}}
\end{Highlighting}
\end{Shaded}

Normal people should use \texttt{\textbackslash{}maketitle} or
\texttt{\textbackslash{}titlepage} instead of using the
\texttt{title\ page} beamer template directly. Beamer already defines
these macros, but we patch them here to make the title page
\texttt{{[}plain{]}} by default and ensure the title frame number
doesn't count.

\begin{description}
\tightlist
\item[\texttt{Inserts}]
the title frame, or causes the current frame to use the
\texttt{title\ page} template.
\end{description}

\begin{Shaded}
\begin{Highlighting}[]
\FunctionTok{\textbackslash{}def\textbackslash{}maketitle}\NormalTok{\{}\CommentTok{\%}
  \FunctionTok{\textbackslash{}ifbeamer@inframe}
    \FunctionTok{\textbackslash{}titlepage}
  \FunctionTok{\textbackslash{}else}
    \FunctionTok{\textbackslash{}begingroup}
    \FunctionTok{\textbackslash{}renewcommand\textbackslash{}footnoterule}\NormalTok{\{\}}\CommentTok{\%}
    \FunctionTok{\textbackslash{}frame}\NormalTok{[plain,noframenumbering]\{}\FunctionTok{\textbackslash{}titlepage}\NormalTok{\}}
    \FunctionTok{\textbackslash{}endgroup}
  \FunctionTok{\textbackslash{}fi}
\NormalTok{\}}
\FunctionTok{\textbackslash{}def\textbackslash{}titlepage}\NormalTok{\{}\CommentTok{\%}
  \CommentTok{\% Apply title{-}page specific footnote settings}
  \FunctionTok{\textbackslash{}renewcommand}\NormalTok{\{}\ExtensionTok{\textbackslash{}@makefntext}\NormalTok{\}[1]\{}\CommentTok{\%}
\NormalTok{    \{}\FunctionTok{\textbackslash{}par\textbackslash{}usebeamercolor}\NormalTok{[fg]\{thanks\}}\FunctionTok{\textbackslash{}usebeamerfont}\NormalTok{\{thanks\}}\SpecialStringTok{$\^{}\{}\SpecialCharTok{\textbackslash{}@}\SpecialStringTok{thefnmark\}$}\NormalTok{\#\#1}\FunctionTok{\textbackslash{}medskip}\NormalTok{\}}\CommentTok{\%}
\NormalTok{  \}}

  \CommentTok{\% Process the title page}
  \FunctionTok{\textbackslash{}usebeamertemplate}\NormalTok{\{title page\}}\FunctionTok{\textbackslash{}@thanks}
\NormalTok{\}}
\end{Highlighting}
\end{Shaded}

\begin{description}
\tightlist
\item[\texttt{title\ graphic}]
Set the title graphic in a zero-height box, so it doesn't change the
position of other elements.
\end{description}

\begin{Shaded}
\begin{Highlighting}[]
\FunctionTok{\textbackslash{}setbeamertemplate}\NormalTok{\{title graphic\}\{}
  \FunctionTok{\textbackslash{}inserttitlegraphic}\CommentTok{\%}
  \FunctionTok{\textbackslash{}par}\CommentTok{\%}
  \FunctionTok{\textbackslash{}vspace*}\NormalTok{\{1em\}}
\NormalTok{\}}
\end{Highlighting}
\end{Shaded}

\begin{description}
\tightlist
\item[\texttt{title}]
Set the title on the title page.
\end{description}

\begin{Shaded}
\begin{Highlighting}[]
\FunctionTok{\textbackslash{}setbeamertemplate}\NormalTok{\{title\}\{}
  \FunctionTok{\textbackslash{}raggedright}\CommentTok{\%}
  \FunctionTok{\textbackslash{}moloch@titleformat}\NormalTok{\{}\FunctionTok{\textbackslash{}inserttitle}\NormalTok{\}}\CommentTok{\%}
  \FunctionTok{\textbackslash{}par}\CommentTok{\%}
\NormalTok{\}}
\end{Highlighting}
\end{Shaded}

\begin{description}
\tightlist
\item[\texttt{subtitle}]
Set the subtitle on the title page.
\end{description}

\begin{Shaded}
\begin{Highlighting}[]
\FunctionTok{\textbackslash{}setbeamertemplate}\NormalTok{\{subtitle\}\{}
  \FunctionTok{\textbackslash{}vspace*}\NormalTok{\{0.3em\}}
  \FunctionTok{\textbackslash{}raggedright}\CommentTok{\%}
  \FunctionTok{\textbackslash{}moloch@subtitleformat}\NormalTok{\{}\FunctionTok{\textbackslash{}insertsubtitle}\NormalTok{\}}\CommentTok{\%}
  \FunctionTok{\textbackslash{}par}\CommentTok{\%}
\NormalTok{\}}
\end{Highlighting}
\end{Shaded}

\begin{description}
\tightlist
\item[\texttt{title\ separator}]
Template to set the title separator.
\end{description}

\begin{Shaded}
\begin{Highlighting}[]
\FunctionTok{\textbackslash{}setbeamertemplate}\NormalTok{\{title separator\}\{}
  \FunctionTok{\textbackslash{}tikzexternaldisable}\CommentTok{\%}
  \KeywordTok{\textbackslash{}begin}\NormalTok{\{}\ExtensionTok{tikzpicture}\NormalTok{\}[baseline=(current bounding box.north)]}
    \FunctionTok{\textbackslash{}mmzUnmemoizable}\CommentTok{\%}
    \FunctionTok{\textbackslash{}fill}\NormalTok{[fg] (0,0) rectangle (}\FunctionTok{\textbackslash{}textwidth}\NormalTok{, }\FunctionTok{\textbackslash{}moloch@titleseparator@linewidth}\NormalTok{);}
    \FunctionTok{\textbackslash{}useasboundingbox}\NormalTok{ (0,0) rectangle (}\FunctionTok{\textbackslash{}textwidth}\NormalTok{,{-}}\FunctionTok{\textbackslash{}moloch@titleseparator@linewidth}\NormalTok{);}
  \KeywordTok{\textbackslash{}end}\NormalTok{\{}\ExtensionTok{tikzpicture}\NormalTok{\}}\CommentTok{\%}
  \FunctionTok{\textbackslash{}tikzexternalenable}\CommentTok{\%}
  \FunctionTok{\textbackslash{}par}\CommentTok{\%}
  \FunctionTok{\textbackslash{}vspace*}\NormalTok{\{0.8em\}}
\NormalTok{\}}
\end{Highlighting}
\end{Shaded}

\begin{description}
\tightlist
\item[\texttt{author}]
Set the author on the title page.
\end{description}

\begin{Shaded}
\begin{Highlighting}[]
\FunctionTok{\textbackslash{}setbeamertemplate}\NormalTok{\{author\}\{}
  \FunctionTok{\textbackslash{}raggedright}\CommentTok{\%}
  \FunctionTok{\textbackslash{}insertauthor}\CommentTok{\%}
  \FunctionTok{\textbackslash{}par}\CommentTok{\%}
  \FunctionTok{\textbackslash{}vspace*}\NormalTok{\{0.5em\}}
\NormalTok{\}}
\end{Highlighting}
\end{Shaded}

\begin{description}
\tightlist
\item[\texttt{institute}]
Set the institute on the title page.
\end{description}

\begin{Shaded}
\begin{Highlighting}[]
\FunctionTok{\textbackslash{}setbeamertemplate}\NormalTok{\{institute\}\{}
  \FunctionTok{\textbackslash{}insertinstitute}\CommentTok{\%}
  \FunctionTok{\textbackslash{}par}\CommentTok{\%}
  \FunctionTok{\textbackslash{}vspace*}\NormalTok{\{1em\}}
\NormalTok{\}}
\end{Highlighting}
\end{Shaded}

\begin{description}
\tightlist
\item[\texttt{date}]
Set the date on the title page.
\end{description}

\begin{Shaded}
\begin{Highlighting}[]
\FunctionTok{\textbackslash{}setbeamertemplate}\NormalTok{\{date\}\{}
  \FunctionTok{\textbackslash{}insertdate}\CommentTok{\%}
  \FunctionTok{\textbackslash{}par}\CommentTok{\%}
\NormalTok{\}}
\end{Highlighting}
\end{Shaded}

\subsection{Section Page}\label{section-page}

\begin{description}
\tightlist
\item[\texttt{section\ Page}]
\end{description}

Template for the section title slide at the beginning of each section.

\begin{Shaded}
\begin{Highlighting}[]
\FunctionTok{\textbackslash{}defbeamertemplate}\NormalTok{\{section page\}\{simple\}\{}
  \KeywordTok{\textbackslash{}begin}\NormalTok{\{}\ExtensionTok{center}\NormalTok{\}}
    \FunctionTok{\textbackslash{}usebeamercolor}\NormalTok{[fg]\{section title\}}
    \FunctionTok{\textbackslash{}usebeamerfont}\NormalTok{\{section title\}}
    \FunctionTok{\textbackslash{}moloch@sectiontitleformat}\NormalTok{\{}\FunctionTok{\textbackslash{}insertsectionhead}\NormalTok{\}}\FunctionTok{\textbackslash{}par}
    \FunctionTok{\textbackslash{}usebeamercolor}\NormalTok{[fg]\{subsection title\}}\CommentTok{\%}
    \FunctionTok{\textbackslash{}usebeamerfont}\NormalTok{\{subsection title\}}\CommentTok{\%}
    \FunctionTok{\textbackslash{}strut}\CommentTok{\%}
    \FunctionTok{\textbackslash{}ifx\textbackslash{}insertsubsectionhead\textbackslash{}@empty\textbackslash{}else}\CommentTok{\%}
      \FunctionTok{\textbackslash{}insertsubsectionhead}\CommentTok{\%}
    \FunctionTok{\textbackslash{}fi}
  \KeywordTok{\textbackslash{}end}\NormalTok{\{}\ExtensionTok{center}\NormalTok{\}}
  \FunctionTok{\textbackslash{}vspace}\NormalTok{\{}\FunctionTok{\textbackslash{}baselineskip}\NormalTok{ {-} 1ex + }\FunctionTok{\textbackslash{}moloch@titleseparator@linewidth}\NormalTok{\}}
\NormalTok{\}}
\FunctionTok{\textbackslash{}defbeamertemplate}\NormalTok{\{section page\}\{progressbar\}\{}
  \FunctionTok{\textbackslash{}centering}
  \KeywordTok{\textbackslash{}begin}\NormalTok{\{}\ExtensionTok{minipage}\NormalTok{\}\{0.7875}\FunctionTok{\textbackslash{}linewidth}\NormalTok{\}}
    \FunctionTok{\textbackslash{}raggedright}
    \FunctionTok{\textbackslash{}usebeamercolor}\NormalTok{[fg]\{section title\}}
    \FunctionTok{\textbackslash{}usebeamerfont}\NormalTok{\{section title\}}
    \FunctionTok{\textbackslash{}moloch@sectiontitleformat}\NormalTok{\{}\FunctionTok{\textbackslash{}insertsectionhead}\NormalTok{\}}\FunctionTok{\textbackslash{}\textbackslash{}}\NormalTok{[{-}0.5}\FunctionTok{\textbackslash{}baselineskip}\NormalTok{]}
    \FunctionTok{\textbackslash{}usebeamertemplate*}\NormalTok{\{progress bar in section page\}}
    \FunctionTok{\textbackslash{}par}
    \FunctionTok{\textbackslash{}usebeamercolor}\NormalTok{[fg]\{subsection title\}}\CommentTok{\%}
    \FunctionTok{\textbackslash{}usebeamerfont}\NormalTok{\{subsection title\}}\CommentTok{\%}
    \FunctionTok{\textbackslash{}strut}\CommentTok{\%}
    \FunctionTok{\textbackslash{}ifx\textbackslash{}insertsubsectionhead\textbackslash{}@empty\textbackslash{}else}\CommentTok{\%}
      \FunctionTok{\textbackslash{}insertsubsectionhead}\CommentTok{\%}
    \FunctionTok{\textbackslash{}fi}
  \KeywordTok{\textbackslash{}end}\NormalTok{\{}\ExtensionTok{minipage}\NormalTok{\}}
  \FunctionTok{\textbackslash{}par}
\NormalTok{\}}
\FunctionTok{\textbackslash{}newcommand}\NormalTok{\{}\ExtensionTok{\textbackslash{}moloch@disablesectionpage}\NormalTok{\}\{}
  \FunctionTok{\textbackslash{}AtBeginSection}\NormalTok{\{}
    \CommentTok{\% intentionally empty}
\NormalTok{  \}}
\NormalTok{\}}
\FunctionTok{\textbackslash{}newcommand}\NormalTok{\{}\ExtensionTok{\textbackslash{}moloch@enablesectionpage}\NormalTok{\}\{}
  \FunctionTok{\textbackslash{}AtBeginSection}\NormalTok{\{}
    \FunctionTok{\textbackslash{}ifbeamer@inframe}
      \FunctionTok{\textbackslash{}sectionpage}
    \FunctionTok{\textbackslash{}else}
      \FunctionTok{\textbackslash{}frame}\NormalTok{[plain,c,noframenumbering]\{}\FunctionTok{\textbackslash{}sectionpage}\NormalTok{\}}
    \FunctionTok{\textbackslash{}fi}
\NormalTok{  \}}
\NormalTok{\}}
\end{Highlighting}
\end{Shaded}

\begin{description}
\tightlist
\item[\texttt{subsection\ page}]
\end{description}

Template for the subsection title slide that can optionally be added to
at the beginning of each subsection.

\begin{Shaded}
\begin{Highlighting}[]
\FunctionTok{\textbackslash{}setbeamertemplate}\NormalTok{\{subsection page\}\{}\CommentTok{\%}
  \FunctionTok{\textbackslash{}usebeamertemplate*}\NormalTok{\{section page\}}
\NormalTok{\}}
\FunctionTok{\textbackslash{}newcommand}\NormalTok{\{}\ExtensionTok{\textbackslash{}moloch@disablesubsectionpage}\NormalTok{\}\{}
  \FunctionTok{\textbackslash{}AtBeginSubsection}\NormalTok{\{}
    \CommentTok{\% intentionally empty}
\NormalTok{  \}}
\NormalTok{\}}
\FunctionTok{\textbackslash{}newcommand}\NormalTok{\{}\ExtensionTok{\textbackslash{}moloch@enablesubsectionpage}\NormalTok{\}\{}
  \FunctionTok{\textbackslash{}AtBeginSubsection}\NormalTok{\{}
    \FunctionTok{\textbackslash{}ifbeamer@inframe}
      \FunctionTok{\textbackslash{}subsectionpage}
    \FunctionTok{\textbackslash{}else}
      \FunctionTok{\textbackslash{}frame}\NormalTok{[plain,c,noframenumbering]\{}\FunctionTok{\textbackslash{}subsectionpage}\NormalTok{\}}
    \FunctionTok{\textbackslash{}fi}
\NormalTok{  \}}
\NormalTok{\}}
\end{Highlighting}
\end{Shaded}

\begin{description}
\tightlist
\item[\texttt{progress\ bar\ in\ section\ page}]
\end{description}

Template for the progress bar displayed by default on the section page.
This code is duplicated in large part in the outer theme's template
\texttt{progress\ bar\ in\ head/foot}.

\begin{Shaded}
\begin{Highlighting}[]
\FunctionTok{\textbackslash{}setbeamertemplate}\NormalTok{\{progress bar in section page\}\{}
  \FunctionTok{\textbackslash{}pgfmathsetlength}\NormalTok{\{}\FunctionTok{\textbackslash{}moloch@progressonsectionpage}\NormalTok{\}\{}
    \FunctionTok{\textbackslash{}textwidth}\NormalTok{ * min(1,}\FunctionTok{\textbackslash{}insertframenumber}\NormalTok{/}\FunctionTok{\textbackslash{}inserttotalframenumber}\NormalTok{)}
\NormalTok{  \}}\CommentTok{\%}
  \FunctionTok{\textbackslash{}tikzexternaldisable}\CommentTok{\%}
  \KeywordTok{\textbackslash{}begin}\NormalTok{\{}\ExtensionTok{tikzpicture}\NormalTok{\}[baseline=(current bounding box.north)]}
    \FunctionTok{\textbackslash{}mmzUnmemoizable}\CommentTok{\%}
    \FunctionTok{\textbackslash{}fill}\NormalTok{[bg]}
\NormalTok{    (0,0)}
\NormalTok{    rectangle}
\NormalTok{    (}\FunctionTok{\textbackslash{}textwidth}\NormalTok{, }\FunctionTok{\textbackslash{}moloch@progressonsectionpage@linewidth}\NormalTok{);}
    \FunctionTok{\textbackslash{}fill}\NormalTok{[fg]}
\NormalTok{    (0,0)}
\NormalTok{    rectangle}
\NormalTok{    (}\FunctionTok{\textbackslash{}moloch@progressonsectionpage}\NormalTok{,}
    \FunctionTok{\textbackslash{}moloch@progressonsectionpage@linewidth}\NormalTok{);}
    \FunctionTok{\textbackslash{}useasboundingbox}\NormalTok{ (0,0) rectangle (}\FunctionTok{\textbackslash{}textwidth}\NormalTok{,{-}}\FunctionTok{\textbackslash{}moloch@progressonsectionpage@linewidth}\NormalTok{);}
  \KeywordTok{\textbackslash{}end}\NormalTok{\{}\ExtensionTok{tikzpicture}\NormalTok{\}}\CommentTok{\%}
  \FunctionTok{\textbackslash{}tikzexternalenable}\CommentTok{\%}
\NormalTok{\}}
\end{Highlighting}
\end{Shaded}

The code above assumes that \texttt{\textbackslash{}insertframenumber}
is less than or equal to
\texttt{\textbackslash{}inserttotalframenumber}. However, this is not
true on the first compile; in the absence of an \texttt{.aux} file,
\texttt{\textbackslash{}inserttotalframenumber} defaults to 1. This
behaviour could cause fatal errors for long presentations, as
\texttt{\textbackslash{}moloch@progressonsectionpage} would exceed TeX's
maximum length (16383.99999pt, roughly 5.75 metres or 18.9 feet). To
avoid this, we increase the default value for
\texttt{\textbackslash{}inserttotalframenumber}; presentations with over
4000 slides will still break on first compile, but users in that
situation likely have deeper problems to solve.

\begin{Shaded}
\begin{Highlighting}[]
\FunctionTok{\textbackslash{}def\textbackslash{}inserttotalframenumber}\NormalTok{\{100\}}
\end{Highlighting}
\end{Shaded}

\subsection{Lists and Floats}\label{lists-and-floats}

Moloch uses custom symbols for the \texttt{itemize} environment. The
symbols are defined as below, using \texttt{pgf} commands to draw the
shapes. This is slightly different than what beamer does by default,
which is to use font glyphs from the math font. But we want to avoid
this reliance on the math font, which may have somewhat surprising side
effects.

By default, we use a filled circle for the first-level \texttt{itemize}
items, a filled square for the second level, and a filled triangle for
the third level. Since the triangle tapers to a point, we add a slight
overhang to it so that it visually aligns better with the other symbols.
We do the same for the circle, but to a lower extent.

\begin{Shaded}
\begin{Highlighting}[]
\FunctionTok{\textbackslash{}newcommand}\NormalTok{\{}\ExtensionTok{\textbackslash{}mitemover}\NormalTok{\}[2]\{}\FunctionTok{\textbackslash{}makebox}\NormalTok{[0pt][r]\{\#1\}}\FunctionTok{\textbackslash{}kern}\NormalTok{{-}\#2\}}
\FunctionTok{\textbackslash{}newcommand}\NormalTok{\{}\ExtensionTok{\textbackslash{}mitem}\NormalTok{\}[1]\{}\FunctionTok{\textbackslash{}mitemover}\NormalTok{\{\#1\}\{0pt\}\}}

\FunctionTok{\textbackslash{}newcommand}\NormalTok{\{}\ExtensionTok{\textbackslash{}overhangSquare}\NormalTok{\}\{0pt\}}
\FunctionTok{\textbackslash{}newcommand}\NormalTok{\{}\ExtensionTok{\textbackslash{}overhangCircle}\NormalTok{\}\{0.05ex\}}
\FunctionTok{\textbackslash{}newcommand}\NormalTok{\{}\ExtensionTok{\textbackslash{}overhangTriangle}\NormalTok{\}\{0.25ex\}}

\FunctionTok{\textbackslash{}newcommand}\NormalTok{\{}\ExtensionTok{\textbackslash{}moloch@circle}\NormalTok{\}\{}
  \KeywordTok{\textbackslash{}begin}\NormalTok{\{}\ExtensionTok{pgfpicture}\NormalTok{\}}
    \FunctionTok{\textbackslash{}pgfsetbaseline}\NormalTok{\{{-}0.7ex\}}
    \FunctionTok{\textbackslash{}pgfpathcircle}\NormalTok{\{}\FunctionTok{\textbackslash{}pgfpoint}\NormalTok{\{0\}\{0\}\}\{0.16em\}}
    \FunctionTok{\textbackslash{}pgfusepath}\NormalTok{\{fill\}}
  \KeywordTok{\textbackslash{}end}\NormalTok{\{}\ExtensionTok{pgfpicture}\NormalTok{\}}\CommentTok{\%}
\NormalTok{\}}
\CommentTok{\%}
\FunctionTok{\textbackslash{}newcommand}\NormalTok{\{}\ExtensionTok{\textbackslash{}moloch@square}\NormalTok{\}\{}
  \KeywordTok{\textbackslash{}begin}\NormalTok{\{}\ExtensionTok{pgfpicture}\NormalTok{\}}
    \FunctionTok{\textbackslash{}pgfsetbaseline}\NormalTok{\{{-}0.7ex\}}
    \FunctionTok{\textbackslash{}pgfpathrectangle}\NormalTok{\{}\FunctionTok{\textbackslash{}pgfpoint}\NormalTok{\{{-}0.165em\}\{{-}0.165em\}\}\{}\FunctionTok{\textbackslash{}pgfpoint}\NormalTok{\{0.31em\}\{0.31em\}\}}
    \FunctionTok{\textbackslash{}pgfusepath}\NormalTok{\{fill\}}
  \KeywordTok{\textbackslash{}end}\NormalTok{\{}\ExtensionTok{pgfpicture}\NormalTok{\}}\CommentTok{\%}
\NormalTok{\}}
\CommentTok{\%}
\FunctionTok{\textbackslash{}newcommand}\NormalTok{\{}\ExtensionTok{\textbackslash{}moloch@triangle}\NormalTok{\}\{}
  \KeywordTok{\textbackslash{}begin}\NormalTok{\{}\ExtensionTok{pgfpicture}\NormalTok{\}}
    \FunctionTok{\textbackslash{}pgfsetbaseline}\NormalTok{\{{-}0.7ex\}}
    \FunctionTok{\textbackslash{}pgfpathmoveto}\NormalTok{\{}\FunctionTok{\textbackslash{}pgfpoint}\NormalTok{\{0.21em\}\{0\}\}       }\CommentTok{\% right vertex (tip)}
    \FunctionTok{\textbackslash{}pgfpathlineto}\NormalTok{\{}\FunctionTok{\textbackslash{}pgfpoint}\NormalTok{\{{-}0.21em\}\{0.21em\}\}  }\CommentTok{\% top left}
    \FunctionTok{\textbackslash{}pgfpathlineto}\NormalTok{\{}\FunctionTok{\textbackslash{}pgfpoint}\NormalTok{\{{-}0.21em\}\{{-}0.21em\}\} }\CommentTok{\% bottom left}
    \FunctionTok{\textbackslash{}pgfpathclose}
    \FunctionTok{\textbackslash{}pgfusepath}\NormalTok{\{fill\}}
  \KeywordTok{\textbackslash{}end}\NormalTok{\{}\ExtensionTok{pgfpicture}\NormalTok{\}}\CommentTok{\%}
\NormalTok{\}}
\end{Highlighting}
\end{Shaded}

Next, we set the itemize templates to use these symbols.

\begin{Shaded}
\begin{Highlighting}[]
\CommentTok{\% \textbackslash{}setbeamertemplate\{itemize item\}[circle]}
\FunctionTok{\textbackslash{}setbeamertemplate}\NormalTok{\{itemize item\}\{}\FunctionTok{\textbackslash{}mitemover}\NormalTok{\{}\FunctionTok{\textbackslash{}moloch@circle}\NormalTok{\}\{}\FunctionTok{\textbackslash{}overhangCircle}\NormalTok{\}\}}
\FunctionTok{\textbackslash{}setbeamertemplate}\NormalTok{\{itemize subitem\}\{}\FunctionTok{\textbackslash{}mitemover}\NormalTok{\{}\FunctionTok{\textbackslash{}moloch@square}\NormalTok{\}\{}\FunctionTok{\textbackslash{}overhangSquare}\NormalTok{\}\}}
\FunctionTok{\textbackslash{}setbeamertemplate}\NormalTok{\{itemize subsubitem\}\{}\FunctionTok{\textbackslash{}mitemover}\NormalTok{\{}\FunctionTok{\textbackslash{}moloch@triangle}\NormalTok{\}\{}\FunctionTok{\textbackslash{}overhangTriangle}\NormalTok{\}\}}
\FunctionTok{\textbackslash{}setbeamertemplate}\NormalTok{\{caption label separator\}\{: \}}
\FunctionTok{\textbackslash{}setbeamertemplate}\NormalTok{\{caption\}[numbered]}
\end{Highlighting}
\end{Shaded}

\subsection{Footnotes}\label{footnotes}

\begin{Shaded}
\begin{Highlighting}[]
\FunctionTok{\textbackslash{}setbeamertemplate}\NormalTok{\{footnote\}\{}\CommentTok{\%}
  \FunctionTok{\textbackslash{}parindent}\NormalTok{ 0em}\FunctionTok{\textbackslash{}noindent\textbackslash{}raggedright} \FunctionTok{\textbackslash{}usebeamercolor}\NormalTok{\{footnote\}}\FunctionTok{\textbackslash{}hbox}\NormalTok{ to}
\NormalTok{  0.8em\{}\FunctionTok{\textbackslash{}hfil\textbackslash{}insertfootnotemark}\NormalTok{\}}\FunctionTok{\textbackslash{}insertfootnotetext}\CommentTok{\%\%\%}
  \FunctionTok{\textbackslash{}par}\CommentTok{\%}
\NormalTok{\}}
\end{Highlighting}
\end{Shaded}

\subsection{Text and Spacing Settings}\label{text-and-spacing-settings}

By default, Beamer frames offer the \texttt{c} option to \emph{almost}
vertically center the text, but the placement is a little too high. To
fix this, we redefine the \texttt{c} option to equalize
\texttt{\textbackslash{}beamer@frametopskip} and
\texttt{\textbackslash{}beamer@framebottomskip}. This solution was
suggested by Enrico Gregorio in an answer to
\href{http://tex.stackexchange.com/questions/247826/}{this Stack
Exchange question}.

\begin{Shaded}
\begin{Highlighting}[]
\FunctionTok{\textbackslash{}define@key}\NormalTok{\{beamerframe\}\{c\}[true]\{}\CommentTok{\% centered}
  \FunctionTok{\textbackslash{}beamer@frametopskip}\NormalTok{=0pt plus 1fil}\FunctionTok{\textbackslash{}relax}\CommentTok{\%}
  \FunctionTok{\textbackslash{}beamer@framebottomskip}\NormalTok{=0pt plus 1fil}\FunctionTok{\textbackslash{}relax}\CommentTok{\%}
  \FunctionTok{\textbackslash{}beamer@frametopskipautobreak}\NormalTok{=0pt plus .4}\FunctionTok{\textbackslash{}paperheight\textbackslash{}relax}\CommentTok{\%}
  \FunctionTok{\textbackslash{}beamer@framebottomskipautobreak}\NormalTok{=0pt plus .6}\FunctionTok{\textbackslash{}paperheight\textbackslash{}relax}\CommentTok{\%}
  \FunctionTok{\textbackslash{}def\textbackslash{}beamer@initfirstlineunskip}\NormalTok{\{\}}\CommentTok{\%}
\NormalTok{\}}
\end{Highlighting}
\end{Shaded}

\subsection{Standout Frames}\label{standout-frames-1}

offers a custom frame format with large, centered text and an inverted
background. To use it, add the key \texttt{standout} to the frame:

\texttt{\textbackslash{}begin\{frame\}{[}standout{]}\ ...\ \textbackslash{}end\{frame\}}.

\begin{description}
\tightlist
\item[\texttt{standout}]
\end{description}

Optional arguments to Beamer's frames are implemented using
\texttt{\textbackslash{}define@key} from the \texttt{keyval} package,
which will execute code when the defined option is called. For the
\texttt{standout} option, we begin a group, change the colors and set
frame options.

\begin{Shaded}
\begin{Highlighting}[]
\FunctionTok{\textbackslash{}providebool}\NormalTok{\{moloch@standout\}}
\FunctionTok{\textbackslash{}define@key}\NormalTok{\{beamerframe\}\{standout\}[true]\{}\CommentTok{\%}
  \FunctionTok{\textbackslash{}booltrue}\NormalTok{\{moloch@standout\}}
  \FunctionTok{\textbackslash{}begingroup}
  \FunctionTok{\textbackslash{}setkeys}\NormalTok{\{beamerframe\}\{c\}}
  \FunctionTok{\textbackslash{}ifbool}\NormalTok{\{moloch@enableStandoutNumbering\}\{\}\{}\CommentTok{\%}
    \FunctionTok{\textbackslash{}setkeys}\NormalTok{\{beamerframe\}\{noframenumbering\}\}}
  \FunctionTok{\textbackslash{}ifbeamercolorempty}\NormalTok{[bg]\{palette primary\}\{}
    \FunctionTok{\textbackslash{}setbeamercolor}\NormalTok{\{background canvas\}\{}
\NormalTok{      use=palette primary,}
\NormalTok{      bg={-}palette primary.fg}
\NormalTok{    \}}
\NormalTok{  \}\{}
    \FunctionTok{\textbackslash{}setbeamercolor}\NormalTok{\{background canvas\}\{}
\NormalTok{      use=palette primary,}
\NormalTok{      bg=palette primary.bg}
\NormalTok{    \}}
\NormalTok{  \}}
  \FunctionTok{\textbackslash{}setbeamercolor}\NormalTok{\{local structure\}\{}
\NormalTok{    fg=palette primary.fg}
\NormalTok{  \}}
  \FunctionTok{\textbackslash{}usebeamercolor}\NormalTok{[fg]\{palette primary\}}
  \FunctionTok{\textbackslash{}setbeamercolor}\NormalTok{\{page number in head/foot\}\{}
\NormalTok{    use=palette primary,}
\NormalTok{    fg=palette primary.fg}
\NormalTok{  \}}
  \FunctionTok{\textbackslash{}ifbool}\NormalTok{\{moloch@enableStandoutFooter\}\{\}\{}\FunctionTok{\textbackslash{}setbeamertemplate}\NormalTok{\{footline\}\{\}\}}
\NormalTok{\}}
\end{Highlighting}
\end{Shaded}

Then we just have to close the group after the standout slide is
finished in order to restore the colours and fonts for the rest of the
presentation. Unfortunately, we cannot use for this (see
\url{http://tex.stackexchange.com/questions/226319/}). Instead, we
prepend the \texttt{\textbackslash{}endgroup} to
\texttt{\textbackslash{}beamer@reseteecodes}, which is run exactly once
at the end of each slide.

\begin{Shaded}
\begin{Highlighting}[]
\FunctionTok{\textbackslash{}pretocmd}\NormalTok{\{}\FunctionTok{\textbackslash{}beamer@reseteecodes}\NormalTok{\}\{}\CommentTok{\%}
  \FunctionTok{\textbackslash{}ifbool}\NormalTok{\{moloch@standout\}\{}
    \FunctionTok{\textbackslash{}endgroup}
    \FunctionTok{\textbackslash{}boolfalse}\NormalTok{\{moloch@standout\}}
\NormalTok{  \}\{\}}
\NormalTok{\}\{\}\{\}}
\end{Highlighting}
\end{Shaded}

We set the fonts and the alignment on the inner content, in such a way
that the speaker's note layout isn't affected by the custom formatting.

\begin{Shaded}
\begin{Highlighting}[]
\FunctionTok{\textbackslash{}AtBeginEnvironment}\NormalTok{\{beamer@frameslide\}\{}
  \FunctionTok{\textbackslash{}ifbool}\NormalTok{\{moloch@standout\}\{}
    \FunctionTok{\textbackslash{}centering}
    \FunctionTok{\textbackslash{}usebeamerfont}\NormalTok{\{standout\}}
\NormalTok{  \}\{\}}
\NormalTok{\}}
\end{Highlighting}
\end{Shaded}

\subsection{Process Package Options}\label{process-package-options-1}

\begin{Shaded}
\begin{Highlighting}[]
\FunctionTok{\textbackslash{}moloch@inner@setdefaults}
\FunctionTok{\textbackslash{}ProcessPgfPackageOptions}\NormalTok{\{/moloch/inner\}}
\end{Highlighting}
\end{Shaded}

\chapter{Outer Theme}\label{outer-theme-1}

A \texttt{beamer} outer theme dictates the style of the frame elements
traditionally set outside the body of each slide: the head, footline,
and frame title.

\subsection{Package dependencies}\label{package-dependencies-2}

\begin{Shaded}
\begin{Highlighting}[]
\FunctionTok{\textbackslash{}RequirePackage}\NormalTok{\{calc\}}
\FunctionTok{\textbackslash{}RequirePackage}\NormalTok{\{pgfopts\}}
\end{Highlighting}
\end{Shaded}

\subsection{Memoization and Tikz
Externalization}\label{memoization-and-tikz-externalization-1}

To avoid generating externalized figures of the progressbar we have to
disable them with ``tikzexternalenable'' and ``tikzexternaldisable''.
However, if the ``external'' library is not loaded we would get
undefined control sequence problems, hence we define them as no-ops if
they are not defined yet. We do the same for the ``mmzUnmemoizable''
command from the memoize package, in order to avoid memoization of the
progress bars.

\begin{Shaded}
\begin{Highlighting}[]
\FunctionTok{\textbackslash{}providecommand}\NormalTok{\{}\ExtensionTok{\textbackslash{}tikzexternalenable}\NormalTok{\}\{\}}
\FunctionTok{\textbackslash{}providecommand}\NormalTok{\{}\ExtensionTok{\textbackslash{}tikzexternaldisable}\NormalTok{\}\{\}}
\FunctionTok{\textbackslash{}providecommand}\NormalTok{\{}\ExtensionTok{\textbackslash{}mmzUnmemoizable}\NormalTok{\}\{\}}
\end{Highlighting}
\end{Shaded}

\subsection{Options}\label{options-2}

\begin{description}
\tightlist
\item[\texttt{progressbar}]
Adds a progress bar to the top, bottom, or frametitle of each slide.
\end{description}

\begin{Shaded}
\begin{Highlighting}[]
\FunctionTok{\textbackslash{}pgfkeys}\NormalTok{\{}
\NormalTok{  /moloch/outer/progressbar/.cd,}
\NormalTok{  .is choice,}
\NormalTok{  none/.code=\{}\CommentTok{\%}
      \FunctionTok{\textbackslash{}setbeamertemplate}\NormalTok{\{headline\}[plain]}
      \FunctionTok{\textbackslash{}setbeamertemplate}\NormalTok{\{frametitle\}[plain]}
      \FunctionTok{\textbackslash{}setbeamertemplate}\NormalTok{\{footline\}[plain]}
\NormalTok{    \},}
\NormalTok{  head/.code=\{}\FunctionTok{\textbackslash{}pgfkeys}\NormalTok{\{/moloch/outer/progressbar=none\}}
      \FunctionTok{\textbackslash{}addtobeamertemplate}\NormalTok{\{headline\}\{\}\{}\CommentTok{\%}
        \FunctionTok{\textbackslash{}usebeamertemplate*}\NormalTok{\{progress bar in head/foot\}}
\NormalTok{      \}}
\NormalTok{    \},}
\NormalTok{  frametitle/.code=\{}\FunctionTok{\textbackslash{}pgfkeys}\NormalTok{\{/moloch/outer/progressbar=none\}}
      \FunctionTok{\textbackslash{}addtobeamertemplate}\NormalTok{\{frametitle\}\{\}\{}\CommentTok{\%}
        \FunctionTok{\textbackslash{}usebeamertemplate*}\NormalTok{\{progress bar in head/foot\}}
\NormalTok{      \}}
\NormalTok{    \},}
\NormalTok{  foot/.code=\{}\FunctionTok{\textbackslash{}pgfkeys}\NormalTok{\{/moloch/outer/progressbar=none\}}
      \FunctionTok{\textbackslash{}addtobeamertemplate}\NormalTok{\{footline\}\{\}\{}\CommentTok{\%}
        \FunctionTok{\textbackslash{}usebeamertemplate*}\NormalTok{\{progress bar in head/foot\}}\CommentTok{\%}
\NormalTok{      \}}
\NormalTok{    \},}
\NormalTok{\}}
\end{Highlighting}
\end{Shaded}

\begin{description}
\tightlist
\item[\texttt{progressbar\ linewidth}]
Sets the linewidth of the progress bar for sectionpages and frames.
\end{description}

\begin{Shaded}
\begin{Highlighting}[]
\FunctionTok{\textbackslash{}newlength}\NormalTok{\{}\FunctionTok{\textbackslash{}moloch@progressonsectionpage}\NormalTok{\}}
\FunctionTok{\textbackslash{}newlength}\NormalTok{\{}\FunctionTok{\textbackslash{}moloch@progressonsectionpage@linewidth}\NormalTok{\}}
\FunctionTok{\textbackslash{}newlength}\NormalTok{\{}\FunctionTok{\textbackslash{}moloch@progressinheadfoot}\NormalTok{\}}
\FunctionTok{\textbackslash{}newlength}\NormalTok{\{}\FunctionTok{\textbackslash{}moloch@progressinheadfoot@linewidth}\NormalTok{\}}
\FunctionTok{\textbackslash{}pgfkeys}\NormalTok{\{}
\NormalTok{  /moloch/outer/.cd,}
\NormalTok{  progressbarlinewidth/.code=\{}
      \FunctionTok{\textbackslash{}setlength}\NormalTok{\{}\FunctionTok{\textbackslash{}moloch@progressonsectionpage@linewidth}\NormalTok{\}\{\#1\}}
      \FunctionTok{\textbackslash{}setlength}\NormalTok{\{}\FunctionTok{\textbackslash{}moloch@progressinheadfoot@linewidth}\NormalTok{\}\{\#1\}}
\NormalTok{    \},}
\NormalTok{  progressbarlinewidth/.default=0.4pt,}
\NormalTok{\}}
\end{Highlighting}
\end{Shaded}

\begin{description}
\tightlist
\item[\texttt{progressbar\ aliases}]
Allows \texttt{progressbar\ linewidth} to be used in
\texttt{\textbackslash{}molochset}.
\end{description}

\begin{Shaded}
\begin{Highlighting}[]
\FunctionTok{\textbackslash{}pgfkeys}\NormalTok{\{}
\NormalTok{  /moloch/outer/.cd,}
\NormalTok{  progressbar linewidth/.code=}\FunctionTok{\textbackslash{}pgfkeysalso}\NormalTok{\{progressbarlinewidth=\#1\},}
\NormalTok{\}}
\end{Highlighting}
\end{Shaded}

\begin{description}
\tightlist
\item[\texttt{frametitle\ margin}]
Sets the margins of the frame title.
\end{description}

\begin{Shaded}
\begin{Highlighting}[]
\FunctionTok{\textbackslash{}pgfkeys}\NormalTok{\{}
\NormalTok{  /moloch/outer/.cd,}
\NormalTok{  frametitlemarginleft/.code=}\FunctionTok{\textbackslash{}renewcommand}\NormalTok{\{}\ExtensionTok{\textbackslash{}moloch@frametitle@margin@left}\NormalTok{\}\{\#1\},}
\NormalTok{  frametitlemarginright/.code=}\FunctionTok{\textbackslash{}renewcommand}\NormalTok{\{}\ExtensionTok{\textbackslash{}moloch@frametitle@margin@right}\NormalTok{\}\{\#1\},}
\NormalTok{  frametitlemargintop/.code=}\FunctionTok{\textbackslash{}renewcommand}\NormalTok{\{}\ExtensionTok{\textbackslash{}moloch@frametitle@margin@top}\NormalTok{\}\{\#1\},}
\NormalTok{  frametitlemarginbottom/.code=}\FunctionTok{\textbackslash{}renewcommand}\NormalTok{\{}\ExtensionTok{\textbackslash{}moloch@frametitle@margin@bottom}\NormalTok{\}\{\#1\},}
\NormalTok{\}}
\end{Highlighting}
\end{Shaded}

\subsection{Deprecated Options}\label{deprecated-options}

These options are deprecated and will be removed in a future version.

\begin{description}
\tightlist
\item[\texttt{numbering}]
Adds slide numbers to the bottom right of each slide.
\end{description}

\begin{Shaded}
\begin{Highlighting}[]
\FunctionTok{\textbackslash{}pgfkeys}\NormalTok{\{}
\NormalTok{  /moloch/outer/numbering/.cd,}
\NormalTok{  .is choice,}
\NormalTok{  none/.code=\{}\CommentTok{\%}
      \FunctionTok{\textbackslash{}PackageWarning}\NormalTok{\{moloch\}\{The \textasciigrave{}\textasciigrave{}numbering\textquotesingle{}\textquotesingle{} option is deprecated.}
\NormalTok{        Use beamer\textquotesingle{}s \textasciigrave{}\textasciigrave{}page number in head/foot\textquotesingle{}\textquotesingle{} template instead\}}\CommentTok{\%}
      \FunctionTok{\textbackslash{}setbeamertemplate}\NormalTok{\{page number in head/foot\}[default]}
\NormalTok{    \},}
\NormalTok{  counter/.code=\{}\CommentTok{\%}
      \FunctionTok{\textbackslash{}PackageWarning}\NormalTok{\{moloch\}\{The \textasciigrave{}\textasciigrave{}numbering\textquotesingle{}\textquotesingle{} option is deprecated.}
\NormalTok{        Use beamer\textquotesingle{}s \textasciigrave{}\textasciigrave{}page number in head/foot\textquotesingle{}\textquotesingle{} template instead\}}\CommentTok{\%}
      \FunctionTok{\textbackslash{}setbeamertemplate}\NormalTok{\{page number in head/foot\}[framenumber]}
\NormalTok{    \},}
\NormalTok{  fraction/.code=\{}\CommentTok{\%}
      \FunctionTok{\textbackslash{}PackageWarning}\NormalTok{\{moloch\}\{The \textasciigrave{}\textasciigrave{}numbering\textquotesingle{}\textquotesingle{} option is deprecated.}
\NormalTok{        Use beamer\textquotesingle{}s \textasciigrave{}\textasciigrave{}page number in head/foot\textquotesingle{}\textquotesingle{} template instead\}}\CommentTok{\%}
      \FunctionTok{\textbackslash{}setbeamertemplate}\NormalTok{\{page number in head/foot\}[totalframenumber]}
\NormalTok{    \},}
\NormalTok{\}}
\end{Highlighting}
\end{Shaded}

\subsection{Slide Numbering}\label{slide-numbering}

Moloch defaults to numbering frames. To modify this, simply copy this
line to your preamble and replace \texttt{framenumber}.

\begin{Shaded}
\begin{Highlighting}[]
\FunctionTok{\textbackslash{}setbeamertemplate}\NormalTok{\{page number in head/foot\}[framenumber]}
\end{Highlighting}
\end{Shaded}

\subsection{Head and footline}\label{head-and-footline}

All good \texttt{beamer} presentations should already remove the
navigation symbols, but removes them automatically (just in case).

\begin{Shaded}
\begin{Highlighting}[]
\FunctionTok{\textbackslash{}setbeamertemplate}\NormalTok{\{navigation symbols\}\{\}}
\end{Highlighting}
\end{Shaded}

\begin{description}
\tightlist
\item[\texttt{headline}]
\end{description}

\begin{description}
\tightlist
\item[\texttt{footline}]
Templates for the head- and footline at the top and bottom of each
frame.
\end{description}

\begin{Shaded}
\begin{Highlighting}[]
\FunctionTok{\textbackslash{}defbeamertemplate}\NormalTok{\{headline\}\{plain\}\{\}}
\FunctionTok{\textbackslash{}defbeamertemplate}\NormalTok{\{footline\}\{plain\}\{}\CommentTok{\%}
  \KeywordTok{\textbackslash{}begin}\NormalTok{\{}\ExtensionTok{beamercolorbox}\NormalTok{\}[}
\NormalTok{      leftskip=4pt,}\CommentTok{\%}
\NormalTok{      rightskip=5pt,}\CommentTok{\%}
\NormalTok{      wd=}\FunctionTok{\textbackslash{}textwidth}\NormalTok{,}\CommentTok{\%}
\NormalTok{    ]\{footline\}}\CommentTok{\%}
    \FunctionTok{\textbackslash{}usebeamercolor}\NormalTok{[fg]\{page number in head/foot\}}\CommentTok{\%}
    \FunctionTok{\textbackslash{}usebeamerfont}\NormalTok{\{page number in head/foot\}}\CommentTok{\%}
    \FunctionTok{\textbackslash{}usebeamertemplate*}\NormalTok{\{frame footer\}}\CommentTok{\%}
    \FunctionTok{\textbackslash{}hfill}\CommentTok{\%}
    \FunctionTok{\textbackslash{}usebeamertemplate*}\NormalTok{\{page number in head/foot\}}\FunctionTok{\textbackslash{}vskip}\NormalTok{4pt}\CommentTok{\%}
  \KeywordTok{\textbackslash{}end}\NormalTok{\{}\ExtensionTok{beamercolorbox}\NormalTok{\}}\CommentTok{\%}
\NormalTok{\}}
\end{Highlighting}
\end{Shaded}

\subsection{Frametitle}\label{frametitle}

\begin{description}
\tightlist
\item[\texttt{frametitle}]
Templates for the frame title, which is optionally underlined with a
progress bar.
\end{description}

\begin{Shaded}
\begin{Highlighting}[]
\FunctionTok{\textbackslash{}newcommand}\NormalTok{\{}\ExtensionTok{\textbackslash{}moloch@frametitlestrut@start}\NormalTok{\}\{}\CommentTok{\%}
  \FunctionTok{\textbackslash{}rule}\NormalTok{\{0pt\}\{}\FunctionTok{\textbackslash{}moloch@frametitle@margin@top}\NormalTok{ + }\FunctionTok{\textbackslash{}ht\textbackslash{}strutbox}\NormalTok{\}}\CommentTok{\%}
\NormalTok{\}}\CommentTok{\%}

\FunctionTok{\textbackslash{}newcommand}\NormalTok{\{}\ExtensionTok{\textbackslash{}moloch@frametitlestrut@end}\NormalTok{\}\{}\CommentTok{\%}
  \FunctionTok{\textbackslash{}rule}\NormalTok{[{-}}\FunctionTok{\textbackslash{}moloch@frametitle@margin@bottom}\NormalTok{]\{0pt\}\{}\FunctionTok{\textbackslash{}moloch@frametitle@margin@bottom}\NormalTok{\}}\CommentTok{\%}
\NormalTok{\}}

\FunctionTok{\textbackslash{}newcommand}\NormalTok{\{}\ExtensionTok{\textbackslash{}moloch@frametitle@margin@left}\NormalTok{\}\{1.6ex\}}
\FunctionTok{\textbackslash{}newcommand}\NormalTok{\{}\ExtensionTok{\textbackslash{}moloch@frametitle@margin@right}\NormalTok{\}\{1.6ex\}}
\FunctionTok{\textbackslash{}newcommand}\NormalTok{\{}\ExtensionTok{\textbackslash{}moloch@frametitle@margin@top}\NormalTok{\}\{1.4ex\}}
\FunctionTok{\textbackslash{}newcommand}\NormalTok{\{}\ExtensionTok{\textbackslash{}moloch@frametitle@margin@bottom}\NormalTok{\}\{1.4ex\}}

\FunctionTok{\textbackslash{}defbeamertemplate}\NormalTok{\{frametitle\}\{plain\}\{}\CommentTok{\%}
  \FunctionTok{\textbackslash{}nointerlineskip}\CommentTok{\%}
  \KeywordTok{\textbackslash{}begin}\NormalTok{\{}\ExtensionTok{beamercolorbox}\NormalTok{\}[}\CommentTok{\%}
\NormalTok{      wd=}\FunctionTok{\textbackslash{}paperwidth}\NormalTok{,}\CommentTok{\%}
\NormalTok{      leftskip=}\FunctionTok{\textbackslash{}moloch@frametitle@margin@left}\NormalTok{,}\CommentTok{\%}
\NormalTok{      rightskip=}\FunctionTok{\textbackslash{}the\textbackslash{}glueexpr} \FunctionTok{\textbackslash{}moloch@frametitle@margin@right}\NormalTok{ plus 1fil}\FunctionTok{\textbackslash{}relax}\NormalTok{,}\CommentTok{\%}
\NormalTok{    ]\{frametitle\}}\CommentTok{\%}
    \FunctionTok{\textbackslash{}usebeamerfont}\NormalTok{\{frametitle\}}\CommentTok{\%}
    \FunctionTok{\textbackslash{}moloch@frametitlestrut@start}\CommentTok{\%}
    \FunctionTok{\textbackslash{}moloch@frametitleformat}\NormalTok{\{}\FunctionTok{\textbackslash{}insertframetitle}\NormalTok{\}}\CommentTok{\%}
\NormalTok{    \{}\CommentTok{\%}
      \FunctionTok{\textbackslash{}ifx\textbackslash{}insertframesubtitle\textbackslash{}@empty}\CommentTok{\%}
        \FunctionTok{\textbackslash{}else}\CommentTok{\%}
\NormalTok{        \{}\CommentTok{\%}
          \FunctionTok{\textbackslash{}par}\CommentTok{\%}
          \FunctionTok{\textbackslash{}usebeamerfont}\NormalTok{\{framesubtitle\}}\CommentTok{\%}
          \FunctionTok{\textbackslash{}vspace}\NormalTok{\{{-}0.8ex\}}\CommentTok{\%}
          \FunctionTok{\textbackslash{}usebeamercolor}\NormalTok{[fg]\{framesubtitle\}}\CommentTok{\%}
          \FunctionTok{\textbackslash{}insertframesubtitle}\CommentTok{\%}
\NormalTok{        \}}\CommentTok{\%}
      \FunctionTok{\textbackslash{}fi}
\NormalTok{    \}}\CommentTok{\%}
    \FunctionTok{\textbackslash{}moloch@frametitlestrut@end}\CommentTok{\%}
  \KeywordTok{\textbackslash{}end}\NormalTok{\{}\ExtensionTok{beamercolorbox}\NormalTok{\}}\CommentTok{\%}
\NormalTok{\}}
\FunctionTok{\textbackslash{}setbeamertemplate}\NormalTok{\{frametitle continuation\}\{}\CommentTok{\%}
  \FunctionTok{\textbackslash{}romannumeral\textbackslash{}insertcontinuationcount}\NormalTok{\}}
\end{Highlighting}
\end{Shaded}

\begin{description}
\tightlist
\item[\texttt{progress\ bar\ in\ head/foot}]
Template for the progress bar optionally displayed below the frame title
on each page. Much of this code is duplicated in the inner theme's
template \texttt{progress\ bar\ in\ section\ page}.
\end{description}

\begin{Shaded}
\begin{Highlighting}[]
\FunctionTok{\textbackslash{}setbeamertemplate}\NormalTok{\{progress bar in head/foot\}\{}
  \FunctionTok{\textbackslash{}nointerlineskip}\CommentTok{\%}
  \FunctionTok{\textbackslash{}pgfmathsetlength}\NormalTok{\{}\FunctionTok{\textbackslash{}moloch@progressinheadfoot}\NormalTok{\}\{}\CommentTok{\%}
    \FunctionTok{\textbackslash{}paperwidth}\NormalTok{ * min(1,}\FunctionTok{\textbackslash{}insertframenumber}\NormalTok{/}\FunctionTok{\textbackslash{}inserttotalframenumber}\NormalTok{)}\CommentTok{\%}
\NormalTok{  \}}\CommentTok{\%}
  \KeywordTok{\textbackslash{}begin}\NormalTok{\{}\ExtensionTok{beamercolorbox}\NormalTok{\}[wd=}\FunctionTok{\textbackslash{}paperwidth}\NormalTok{]\{progress bar in head/foot\}}
    \FunctionTok{\textbackslash{}tikzexternaldisable}\CommentTok{\%}
    \KeywordTok{\textbackslash{}begin}\NormalTok{\{}\ExtensionTok{tikzpicture}\NormalTok{\}}
      \FunctionTok{\textbackslash{}mmzUnmemoizable}\CommentTok{\%}
      \FunctionTok{\textbackslash{}fill}\NormalTok{[bg]}
\NormalTok{      (0,0)}
\NormalTok{      rectangle}
\NormalTok{      (}\FunctionTok{\textbackslash{}paperwidth}\NormalTok{, }\FunctionTok{\textbackslash{}moloch@progressinheadfoot@linewidth}\NormalTok{);}
      \FunctionTok{\textbackslash{}fill}\NormalTok{[fg]}
\NormalTok{      (0,0)}
\NormalTok{      rectangle}
\NormalTok{      (}\FunctionTok{\textbackslash{}moloch@progressinheadfoot}\NormalTok{, }\FunctionTok{\textbackslash{}moloch@progressinheadfoot@linewidth}\NormalTok{);}
    \KeywordTok{\textbackslash{}end}\NormalTok{\{}\ExtensionTok{tikzpicture}\NormalTok{\}}
    \FunctionTok{\textbackslash{}tikzexternalenable}\CommentTok{\%}
  \KeywordTok{\textbackslash{}end}\NormalTok{\{}\ExtensionTok{beamercolorbox}\NormalTok{\}}
\NormalTok{\}}
\end{Highlighting}
\end{Shaded}

\subsection{Process package options}\label{process-package-options-2}

\begin{Shaded}
\begin{Highlighting}[]
\FunctionTok{\textbackslash{}moloch@outer@setdefaults}
\FunctionTok{\textbackslash{}ProcessPgfPackageOptions}\NormalTok{\{/moloch/outer\}}
\end{Highlighting}
\end{Shaded}

\chapter{Font Theme}\label{font-theme-1}

A \texttt{beamer} font theme sets the style of the font used in the
document.

\subsection{Package dependencies}\label{package-dependencies-3}

\begin{Shaded}
\begin{Highlighting}[]
\FunctionTok{\textbackslash{}RequirePackage}\NormalTok{\{etoolbox\}}
\FunctionTok{\textbackslash{}RequirePackage}\NormalTok{\{pgfopts\}}
\end{Highlighting}
\end{Shaded}

\subsection{General font definitions}\label{general-font-definitions}

\begin{Shaded}
\begin{Highlighting}[]
\FunctionTok{\textbackslash{}setbeamerfont}\NormalTok{\{title\}\{size=}\FunctionTok{\textbackslash{}Large}\NormalTok{, series=}\FunctionTok{\textbackslash{}bfseries}\NormalTok{\}}
\FunctionTok{\textbackslash{}setbeamerfont}\NormalTok{\{author\}\{size=}\FunctionTok{\textbackslash{}small}\NormalTok{\}}
\FunctionTok{\textbackslash{}setbeamerfont}\NormalTok{\{date\}\{size=}\FunctionTok{\textbackslash{}small}\NormalTok{\}}
\FunctionTok{\textbackslash{}setbeamerfont}\NormalTok{\{section title\}\{size=}\FunctionTok{\textbackslash{}Large}\NormalTok{, series=}\FunctionTok{\textbackslash{}bfseries}\NormalTok{\}}
\FunctionTok{\textbackslash{}setbeamerfont}\NormalTok{\{block title\}\{size=}\FunctionTok{\textbackslash{}normalsize}\NormalTok{, series=}\FunctionTok{\textbackslash{}bfseries}\NormalTok{\}}
\FunctionTok{\textbackslash{}setbeamerfont}\NormalTok{\{block title alerted\}\{size=}\FunctionTok{\textbackslash{}normalsize}\NormalTok{, series=}\FunctionTok{\textbackslash{}bfseries}\NormalTok{\}}
\FunctionTok{\textbackslash{}setbeamerfont*}\NormalTok{\{subtitle\}\{size=}\FunctionTok{\textbackslash{}large}\NormalTok{\}}
\FunctionTok{\textbackslash{}setbeamerfont}\NormalTok{\{frametitle\}\{size=}\FunctionTok{\textbackslash{}large}\NormalTok{, series=}\FunctionTok{\textbackslash{}bfseries}\NormalTok{\}}
\FunctionTok{\textbackslash{}setbeamerfont}\NormalTok{\{framesubtitle\}\{size=}\FunctionTok{\textbackslash{}small}\NormalTok{\}}
\FunctionTok{\textbackslash{}setbeamerfont}\NormalTok{\{caption\}\{size=}\FunctionTok{\textbackslash{}small}\NormalTok{\}}
\FunctionTok{\textbackslash{}setbeamerfont}\NormalTok{\{caption name\}\{series=}\FunctionTok{\textbackslash{}bfseries}\NormalTok{\}}
\FunctionTok{\textbackslash{}setbeamerfont}\NormalTok{\{description item\}\{series=}\FunctionTok{\textbackslash{}bfseries}\NormalTok{\}}
\FunctionTok{\textbackslash{}setbeamerfont}\NormalTok{\{standout\}\{size=}\FunctionTok{\textbackslash{}Large}\NormalTok{, series=}\FunctionTok{\textbackslash{}bfseries}\NormalTok{\}}
\end{Highlighting}
\end{Shaded}

\subsection{Title format options}\label{title-format-options}

\begin{description}
\tightlist
\item[\texttt{titleformat\ title}]
Controls the format of the title.
\end{description}

\begin{Shaded}
\begin{Highlighting}[]
\FunctionTok{\textbackslash{}pgfkeys}\NormalTok{\{}
\NormalTok{  /moloch/font/titleformat title/.cd,}
\NormalTok{  .is choice,}
\NormalTok{  regular/.code=\{}\CommentTok{\%}
      \FunctionTok{\textbackslash{}let\textbackslash{}moloch@titleformat\textbackslash{}@empty}\CommentTok{\%}
      \FunctionTok{\textbackslash{}setbeamerfont}\NormalTok{\{title\}\{shape=}\FunctionTok{\textbackslash{}normalfont}\NormalTok{\}}\CommentTok{\%}
\NormalTok{    \},}
\NormalTok{  smallcaps/.code=\{}\CommentTok{\%}
      \FunctionTok{\textbackslash{}let\textbackslash{}moloch@titleformat\textbackslash{}@empty}\CommentTok{\%}
      \FunctionTok{\textbackslash{}setbeamerfont}\NormalTok{\{title\}\{shape=}\FunctionTok{\textbackslash{}scshape}\NormalTok{\}}\CommentTok{\%}
\NormalTok{    \},}
\NormalTok{  allsmallcaps/.code=\{}\CommentTok{\%}
      \FunctionTok{\textbackslash{}let\textbackslash{}moloch@titleformat\textbackslash{}lowercase}\CommentTok{\%}
      \FunctionTok{\textbackslash{}setbeamerfont}\NormalTok{\{title\}\{shape=}\FunctionTok{\textbackslash{}scshape}\NormalTok{\}}\CommentTok{\%}
      \FunctionTok{\textbackslash{}PackageNote}\NormalTok{\{beamerthememoloch\}\{}\CommentTok{\%}
\NormalTok{        Be aware that titleformat title=allsmallcaps can}
\NormalTok{        lead to problems\}}
\NormalTok{    \},}
\NormalTok{  allcaps/.code=\{}\CommentTok{\%}
      \FunctionTok{\textbackslash{}let\textbackslash{}moloch@titleformat\textbackslash{}uppercase}\CommentTok{\%}
      \FunctionTok{\textbackslash{}setbeamerfont}\NormalTok{\{title\}\{shape=}\FunctionTok{\textbackslash{}normalfont}\NormalTok{\}}
      \FunctionTok{\textbackslash{}PackageNote}\NormalTok{\{beamerthememoloch\}\{}\CommentTok{\%}
\NormalTok{        Be aware that titleformat title=allcaps can lead to problems}\CommentTok{\%}
\NormalTok{      \}}
\NormalTok{    \},}
\NormalTok{\}}
\end{Highlighting}
\end{Shaded}

\begin{description}
\tightlist
\item[\texttt{titleformat\ subtitle}]
Control the format of the subtitle.
\end{description}

\begin{Shaded}
\begin{Highlighting}[]
\FunctionTok{\textbackslash{}pgfkeys}\NormalTok{\{}
\NormalTok{  /moloch/font/titleformat subtitle/.cd,}
\NormalTok{  .is choice,}
\NormalTok{  regular/.code=\{}\CommentTok{\%}
      \FunctionTok{\textbackslash{}let\textbackslash{}moloch@subtitleformat\textbackslash{}@empty}\CommentTok{\%}
      \FunctionTok{\textbackslash{}setbeamerfont}\NormalTok{\{subtitle\}\{shape=}\FunctionTok{\textbackslash{}normalfont}\NormalTok{\}}\CommentTok{\%}
\NormalTok{    \},}
\NormalTok{  smallcaps/.code=\{}\CommentTok{\%}
      \FunctionTok{\textbackslash{}let\textbackslash{}moloch@subtitleformat\textbackslash{}@empty}\CommentTok{\%}
      \FunctionTok{\textbackslash{}setbeamerfont}\NormalTok{\{subtitle\}\{shape=}\FunctionTok{\textbackslash{}scshape}\NormalTok{\}}\CommentTok{\%}
\NormalTok{    \},}
\NormalTok{  allsmallcaps/.code=\{}\CommentTok{\%}
      \FunctionTok{\textbackslash{}let\textbackslash{}moloch@subtitleformat\textbackslash{}MakeLowercase}\CommentTok{\%}
      \FunctionTok{\textbackslash{}setbeamerfont}\NormalTok{\{subtitle\}\{shape=}\FunctionTok{\textbackslash{}scshape}\NormalTok{\}}\CommentTok{\%}
      \FunctionTok{\textbackslash{}PackageNote}\NormalTok{\{beamerthememoloch\}\{}\CommentTok{\%}
\NormalTok{        Be aware that titleformat subtitle=allsmallcaps}
\NormalTok{        can lead to problems\}}
\NormalTok{    \},}
\NormalTok{  allcaps/.code=\{}\CommentTok{\%}
      \FunctionTok{\textbackslash{}let\textbackslash{}moloch@subtitleformat\textbackslash{}MakeUppercase}\CommentTok{\%}
      \FunctionTok{\textbackslash{}setbeamerfont}\NormalTok{\{subtitle\}\{shape=}\FunctionTok{\textbackslash{}normalfont}\NormalTok{\}}\CommentTok{\%}
      \FunctionTok{\textbackslash{}PackageNote}\NormalTok{\{beamerthememoloch\}\{}\CommentTok{\%}
\NormalTok{        Be aware that titleformat subtitle=allcaps can}
\NormalTok{        lead to problems\}}
\NormalTok{    \},}
\NormalTok{\}}
\end{Highlighting}
\end{Shaded}

\begin{description}
\tightlist
\item[\texttt{titleformat\ section}]
Controls the format of the section title.
\end{description}

\begin{Shaded}
\begin{Highlighting}[]
\FunctionTok{\textbackslash{}pgfkeys}\NormalTok{\{}
\NormalTok{  /moloch/font/titleformat section/.cd,}
\NormalTok{  .is choice,}
\NormalTok{  regular/.code=\{}\CommentTok{\%}
      \FunctionTok{\textbackslash{}let\textbackslash{}moloch@sectiontitleformat\textbackslash{}@empty}\CommentTok{\%}
      \FunctionTok{\textbackslash{}setbeamerfont}\NormalTok{\{section title\}\{shape=}\FunctionTok{\textbackslash{}normalfont}\NormalTok{\}}\CommentTok{\%}
\NormalTok{    \},}
\NormalTok{  smallcaps/.code=\{}\CommentTok{\%}
      \FunctionTok{\textbackslash{}let\textbackslash{}moloch@sectiontitleformat\textbackslash{}@empty}\CommentTok{\%}
      \FunctionTok{\textbackslash{}setbeamerfont}\NormalTok{\{section title\}\{shape=}\FunctionTok{\textbackslash{}scshape}\NormalTok{\}}\CommentTok{\%}
\NormalTok{    \},}
\NormalTok{  allsmallcaps/.code=\{}\CommentTok{\%}
      \FunctionTok{\textbackslash{}let\textbackslash{}moloch@sectiontitleformat\textbackslash{}MakeLowercase}\CommentTok{\%}
      \FunctionTok{\textbackslash{}setbeamerfont}\NormalTok{\{section title\}\{shape=}\FunctionTok{\textbackslash{}scshape}\NormalTok{\}}\CommentTok{\%}
      \FunctionTok{\textbackslash{}PackageNote}\NormalTok{\{beamerthememoloch\}\{}\CommentTok{\%}
\NormalTok{        Be aware that titleformat section=allsmallcaps}
\NormalTok{        can lead to problems\}}
\NormalTok{    \},}
\NormalTok{  allcaps/.code=\{}\CommentTok{\%}
      \FunctionTok{\textbackslash{}let\textbackslash{}moloch@sectiontitleformat\textbackslash{}MakeUppercase}\CommentTok{\%}
      \FunctionTok{\textbackslash{}setbeamerfont}\NormalTok{\{section title\}\{shape=}\FunctionTok{\textbackslash{}normalfont}\NormalTok{\}}\CommentTok{\%}
      \FunctionTok{\textbackslash{}PackageNote}\NormalTok{\{beamerthememoloch\}\{}\CommentTok{\%}
\NormalTok{        Be aware that titleformat section=allcaps}
\NormalTok{        can lead to problems\}}
\NormalTok{    \},}
\NormalTok{\}}
\end{Highlighting}
\end{Shaded}

\begin{description}
\tightlist
\item[\texttt{frametitleformat}]
Control the format of the frame title.
\end{description}

\begin{Shaded}
\begin{Highlighting}[]
\FunctionTok{\textbackslash{}pgfkeys}\NormalTok{\{}
\NormalTok{  /moloch/font/titleformat frame/.cd,}
\NormalTok{  .is choice,}
\NormalTok{  regular/.code=\{}\CommentTok{\%}
      \FunctionTok{\textbackslash{}let\textbackslash{}moloch@frametitleformat\textbackslash{}@empty}\CommentTok{\%}
      \FunctionTok{\textbackslash{}setbeamerfont}\NormalTok{\{frametitle\}\{shape=}\FunctionTok{\textbackslash{}normalfont}\NormalTok{\}}\CommentTok{\%}
\NormalTok{    \},}
\NormalTok{  smallcaps/.code=\{}\CommentTok{\%}
      \FunctionTok{\textbackslash{}let\textbackslash{}moloch@frametitleformat\textbackslash{}@empty}\CommentTok{\%}
      \FunctionTok{\textbackslash{}setbeamerfont}\NormalTok{\{frametitle\}\{shape=}\FunctionTok{\textbackslash{}scshape}\NormalTok{\}}\CommentTok{\%}
\NormalTok{    \},}
\NormalTok{  allsmallcaps/.code=\{}\CommentTok{\%}
      \FunctionTok{\textbackslash{}let\textbackslash{}moloch@frametitleformat\textbackslash{}MakeLowercase}\CommentTok{\%}
      \FunctionTok{\textbackslash{}setbeamerfont}\NormalTok{\{frametitle\}\{shape=}\FunctionTok{\textbackslash{}scshape}\NormalTok{\}}\CommentTok{\%}
      \FunctionTok{\textbackslash{}PackageNote}\NormalTok{\{beamerthememoloch\}\{}\CommentTok{\%}
\NormalTok{        Be aware that titleformat frame=allsmallcaps}
\NormalTok{        can lead to problems\}}
\NormalTok{    \},}
\NormalTok{  allcaps/.code=\{}\CommentTok{\%}
      \FunctionTok{\textbackslash{}let\textbackslash{}moloch@frametitleformat\textbackslash{}MakeUppercase}\CommentTok{\%}
      \FunctionTok{\textbackslash{}setbeamerfont}\NormalTok{\{frametitle\}\{shape=}\FunctionTok{\textbackslash{}normalfont}\NormalTok{\}}
      \FunctionTok{\textbackslash{}PackageNote}\NormalTok{\{beamerthememoloch\}\{}\CommentTok{\%}
\NormalTok{        Be aware that titleformat frame=allcaps can lead to problems}\CommentTok{\%}
\NormalTok{      \}}
\NormalTok{    \},}
\NormalTok{\}}
\end{Highlighting}
\end{Shaded}

\begin{description}
\tightlist
\item[\texttt{titleformat\ aliases}]
Allows \texttt{titleformat\ title} et al.~to be used in the
\texttt{\textbackslash{}usetheme} declaration, where LaTeX automatically
removes all spaces.
\end{description}

\begin{Shaded}
\begin{Highlighting}[]
\FunctionTok{\textbackslash{}pgfkeys}\NormalTok{\{}
\NormalTok{  /moloch/font/.cd,}
\NormalTok{  titleformattitle/.code=}\FunctionTok{\textbackslash{}pgfkeysalso}\NormalTok{\{titleformat title=\#1\},}
\NormalTok{  titleformatsubtitle/.code=}\FunctionTok{\textbackslash{}pgfkeysalso}\NormalTok{\{titleformat subtitle=\#1\},}
\NormalTok{  titleformatsection/.code=}\FunctionTok{\textbackslash{}pgfkeysalso}\NormalTok{\{titleformat section=\#1\},}
\NormalTok{  titleformatframe/.code=}\FunctionTok{\textbackslash{}pgfkeysalso}\NormalTok{\{titleformat frame=\#1\},}
\NormalTok{\}}
\end{Highlighting}
\end{Shaded}

\begin{description}
\tightlist
\item[\texttt{@font@setdefaults}]
Sets default values for font theme options.
\end{description}

\begin{Shaded}
\begin{Highlighting}[]
\FunctionTok{\textbackslash{}newcommand}\NormalTok{\{}\ExtensionTok{\textbackslash{}moloch@font@setdefaults}\NormalTok{\}\{}
  \FunctionTok{\textbackslash{}pgfkeys}\NormalTok{\{/moloch/font/.cd,}
\NormalTok{    titleformat title=regular,}
\NormalTok{    titleformat subtitle=regular,}
\NormalTok{    titleformat section=regular,}
\NormalTok{    titleformat frame=regular,}
\NormalTok{  \}}
\NormalTok{\}}
\end{Highlighting}
\end{Shaded}

We first define hooks to change the case format of the titles.

\begin{Shaded}
\begin{Highlighting}[]
\FunctionTok{\textbackslash{}def\textbackslash{}moloch@titleformat}\NormalTok{\#1\{\#1\}}
\FunctionTok{\textbackslash{}def\textbackslash{}moloch@subtitleformat}\NormalTok{\#1\{\#1\}}
\FunctionTok{\textbackslash{}def\textbackslash{}moloch@sectiontitleformat}\NormalTok{\#1\{\#1\}}
\FunctionTok{\textbackslash{}def\textbackslash{}moloch@frametitleformat}\NormalTok{\#1\{\#1\}}
\end{Highlighting}
\end{Shaded}

To make the uppercase and lowercase macros work in the title, subtitle,
etc., we have to patch the appropriate \texttt{beamer} commands that set
their values. This solution was suggested by Enrico Gregorio in an
answer to \href{http://tex.stackexchange.com/questions/112526/}{this
StackExchange question}.

\subsection{Process package options}\label{process-package-options-3}

\begin{Shaded}
\begin{Highlighting}[]
\FunctionTok{\textbackslash{}moloch@font@setdefaults}
\FunctionTok{\textbackslash{}ProcessPgfPackageOptions}\NormalTok{\{/moloch/font\}}
\end{Highlighting}
\end{Shaded}

\chapter{Color Theme}\label{color-theme-1}

\subsection{Package Dependencies}\label{package-dependencies-4}

\begin{Shaded}
\begin{Highlighting}[]
\FunctionTok{\textbackslash{}RequirePackage}\NormalTok{\{pgfopts\}}
\end{Highlighting}
\end{Shaded}

\subsection{Options}\label{options-3}

\begin{description}
\tightlist
\item[\texttt{block}]
Optionally adds a light grey background to block environments like
\texttt{theorem} and \texttt{example}.
\end{description}

\begin{Shaded}
\begin{Highlighting}[]
\FunctionTok{\textbackslash{}pgfkeys}\NormalTok{\{}
\NormalTok{  /moloch/color/block/.cd,}
\NormalTok{  .is choice,}
\NormalTok{  transparent/.code=}\FunctionTok{\textbackslash{}moloch@block@transparent}\NormalTok{,}
\NormalTok{  fill/.code=}\FunctionTok{\textbackslash{}moloch@block@fill}\NormalTok{,}
\NormalTok{\}}
\end{Highlighting}
\end{Shaded}

\begin{description}
\tightlist
\item[\texttt{colors}]
Provides the option to have a dark background and light foreground
instead of the reverse.
\end{description}

\begin{Shaded}
\begin{Highlighting}[]
\FunctionTok{\textbackslash{}pgfkeys}\NormalTok{\{}
\NormalTok{  /moloch/color/background/.cd,}
\NormalTok{  .is choice,}
\NormalTok{  dark/.code=}\FunctionTok{\textbackslash{}moloch@colors@dark}\NormalTok{,}
\NormalTok{  light/.code=}\FunctionTok{\textbackslash{}moloch@colors@light}\NormalTok{,}
\NormalTok{\}}
\end{Highlighting}
\end{Shaded}

\begin{description}
\tightlist
\item[\texttt{@color@setdefaults}]
Sets default values for color theme options.
\end{description}

\begin{Shaded}
\begin{Highlighting}[]
\FunctionTok{\textbackslash{}newcommand}\NormalTok{\{}\ExtensionTok{\textbackslash{}moloch@color@setdefaults}\NormalTok{\}\{}
  \FunctionTok{\textbackslash{}pgfkeys}\NormalTok{\{/moloch/color/.cd,}
\NormalTok{    background=light,}
\NormalTok{  \}}
\NormalTok{\}}
\end{Highlighting}
\end{Shaded}

\subsection{Base Colors}\label{base-colors}

\begin{Shaded}
\begin{Highlighting}[]
\FunctionTok{\textbackslash{}definecolor}\NormalTok{\{mDarkBrown\}\{HTML\}\{604c38\}}
\FunctionTok{\textbackslash{}definecolor}\NormalTok{\{mDarkTeal\}\{HTML\}\{23373b\}}
\FunctionTok{\textbackslash{}definecolor}\NormalTok{\{mLightBrown\}\{HTML\}\{EB811B\}}
\FunctionTok{\textbackslash{}definecolor}\NormalTok{\{mLightGreen\}\{RGB\}\{0,128,128\}}
\end{Highlighting}
\end{Shaded}

\subsection{Base Styles}\label{base-styles}

All colors in are derived from the definitions of \texttt{normal\ text},
\texttt{alerted\ text}, and \texttt{example\ text}.

\begin{Shaded}
\begin{Highlighting}[]
\FunctionTok{\textbackslash{}setbeamercolor}\NormalTok{\{alerted text\}\{}\CommentTok{\%}
\NormalTok{  fg=mLightBrown}
\NormalTok{\}}
\FunctionTok{\textbackslash{}setbeamercolor}\NormalTok{\{example text\}\{}\CommentTok{\%}
\NormalTok{  fg=mLightGreen}
\NormalTok{\}}
\end{Highlighting}
\end{Shaded}

\subsection{Hooks for Color Themes}\label{hooks-for-color-themes}

Moloch color themes can register light and dark color schemes using the
commands below. The registered colors will be stored in the macros
\texttt{\textbackslash{}moloch@define@light@colors} and
\texttt{\textbackslash{}moloch@define@dark@colors} respectively. These
macros are invoked when the \texttt{background=light} or
\texttt{background=dark} options are selected.

\begin{Shaded}
\begin{Highlighting}[]
\FunctionTok{\textbackslash{}newcommand}\NormalTok{\{}\ExtensionTok{\textbackslash{}moloch@define@light@colors}\NormalTok{\}\{\}}
\FunctionTok{\textbackslash{}newcommand}\NormalTok{\{}\ExtensionTok{\textbackslash{}moloch@define@dark@colors}\NormalTok{\}\{\}}

\FunctionTok{\textbackslash{}newcommand}\NormalTok{\{}\ExtensionTok{\textbackslash{}moloch@register@light@colors}\NormalTok{\}[1]\{}\CommentTok{\%}
  \FunctionTok{\textbackslash{}renewcommand}\NormalTok{\{}\ExtensionTok{\textbackslash{}moloch@define@light@colors}\NormalTok{\}\{\#1\}}\CommentTok{\%}
\NormalTok{\}}

\FunctionTok{\textbackslash{}newcommand}\NormalTok{\{}\ExtensionTok{\textbackslash{}moloch@register@dark@colors}\NormalTok{\}[1]\{}\CommentTok{\%}
  \FunctionTok{\textbackslash{}renewcommand}\NormalTok{\{}\ExtensionTok{\textbackslash{}moloch@define@dark@colors}\NormalTok{\}\{\#1\}}\CommentTok{\%}
\NormalTok{\}}

\FunctionTok{\textbackslash{}newcommand}\NormalTok{\{}\ExtensionTok{\textbackslash{}moloch@colors@dark}\NormalTok{\}\{}
  \FunctionTok{\textbackslash{}moloch@define@dark@colors}
  \FunctionTok{\textbackslash{}usebeamercolor}\NormalTok{[fg]\{normal text\}}
\NormalTok{\}}
\FunctionTok{\textbackslash{}newcommand}\NormalTok{\{}\ExtensionTok{\textbackslash{}moloch@colors@light}\NormalTok{\}\{}
  \FunctionTok{\textbackslash{}moloch@define@light@colors}
  \FunctionTok{\textbackslash{}usebeamercolor}\NormalTok{[fg]\{normal text\}}
\NormalTok{\}}

\FunctionTok{\textbackslash{}moloch@register@light@colors}\NormalTok{\{}\CommentTok{\%}
  \FunctionTok{\textbackslash{}setbeamercolor}\NormalTok{\{normal text\}\{}\CommentTok{\%}
\NormalTok{    fg=mDarkTeal,}
\NormalTok{    bg=black!2}
\NormalTok{  \}}\CommentTok{\%}
\NormalTok{\}}

\FunctionTok{\textbackslash{}moloch@register@dark@colors}\NormalTok{\{}\CommentTok{\%}
  \FunctionTok{\textbackslash{}setbeamercolor}\NormalTok{\{normal text\}\{}\CommentTok{\%}
\NormalTok{    fg=black!2,}
\NormalTok{    bg=mDarkTeal}
\NormalTok{  \}}\CommentTok{\%}
\NormalTok{\}}
\end{Highlighting}
\end{Shaded}

\subsection{Derived Colors}\label{derived-colors}

The titles and structural elements (e.g.~\texttt{itemize} bullets) are
set in the same color as \texttt{normal\ text}. This would ideally done
by setting \texttt{normal\ text} as a parent style, which we do to set
\texttt{titlelike}, but this doesn't work for \texttt{structure} as its
foreground is set explicitly in \texttt{beamercolorthemedefault.sty}.

\begin{Shaded}
\begin{Highlighting}[]
\FunctionTok{\textbackslash{}setbeamercolor}\NormalTok{\{titlelike\}\{use=normal text, parent=normal text\}}
\FunctionTok{\textbackslash{}setbeamercolor}\NormalTok{\{author\}\{use=normal text, parent=normal text\}}
\FunctionTok{\textbackslash{}setbeamercolor}\NormalTok{\{date\}\{use=normal text, parent=normal text\}}
\FunctionTok{\textbackslash{}setbeamercolor}\NormalTok{\{institute\}\{}\CommentTok{\%}
\NormalTok{  use=normal text, fg=normal text.fg!80!normal text.bg\}}
\FunctionTok{\textbackslash{}setbeamercolor}\NormalTok{\{structure\}\{use=normal text, fg=normal text.fg\}}
\FunctionTok{\textbackslash{}setbeamercolor}\NormalTok{\{thanks\}\{}\CommentTok{\%}
\NormalTok{  use=normal text,fg=normal text.fg!80!normal text.bg\}}
\end{Highlighting}
\end{Shaded}

The ``primary'' palette should be used for the most important
navigational elements, and possibly of other elements. uses it for frame
titles and slides.

\begin{Shaded}
\begin{Highlighting}[]
\FunctionTok{\textbackslash{}setbeamercolor}\NormalTok{\{palette primary\}\{}\CommentTok{\%}
\NormalTok{  use=normal text,}
\NormalTok{  fg=normal text.bg,}
\NormalTok{  bg=normal text.fg}
\NormalTok{\}}
\FunctionTok{\textbackslash{}setbeamercolor}\NormalTok{\{frametitle\}\{}\CommentTok{\%}
\NormalTok{  use=palette primary,}
\NormalTok{  parent=palette primary}
\NormalTok{\}}
\end{Highlighting}
\end{Shaded}

The inner or outer themes optionally display progress bars in various
locations. Their color is set by \texttt{progress\ bar} but the two
different kinds can be customized separately. The horizontal rule on the
title page is also set based on the progress bar color and can be
customized with \texttt{title\ separator}.

\begin{Shaded}
\begin{Highlighting}[]
\FunctionTok{\textbackslash{}setbeamercolor}\NormalTok{\{progress bar\}\{}\CommentTok{\%}
\NormalTok{  use=alerted text,}
\NormalTok{  fg=alerted text.fg,}
\NormalTok{  bg=alerted text.fg!50!black!30}
\NormalTok{\}}
\FunctionTok{\textbackslash{}setbeamercolor}\NormalTok{\{title separator\}\{}
\NormalTok{  use=progress bar,}
\NormalTok{  parent=progress bar}
\NormalTok{\}}
\FunctionTok{\textbackslash{}setbeamercolor}\NormalTok{\{progress bar in head/foot\}\{}\CommentTok{\%}
\NormalTok{  use=progress bar,}
\NormalTok{  parent=progress bar}
\NormalTok{\}}
\FunctionTok{\textbackslash{}setbeamercolor}\NormalTok{\{progress bar in section page\}\{}
\NormalTok{  use=progress bar,}
\NormalTok{  parent=progress bar}
\NormalTok{\}}
\end{Highlighting}
\end{Shaded}

Block environments such as \texttt{theorem} and \texttt{example} have no
background color by default. The option \texttt{block=fill} sets a
background color based on the background and foreground of
\texttt{normal\ text}. The option \texttt{block=transparent} reverts the
block environments to an empty background, which can be useful if
changing colors mid-presentation.

\begin{Shaded}
\begin{Highlighting}[]
\FunctionTok{\textbackslash{}newcommand}\NormalTok{\{}\ExtensionTok{\textbackslash{}moloch@block@transparent}\NormalTok{\}\{}
  \FunctionTok{\textbackslash{}setbeamercolor}\NormalTok{\{block title\}\{bg=\}}
  \FunctionTok{\textbackslash{}setbeamercolor}\NormalTok{\{block body\}\{bg=\}}
  \FunctionTok{\textbackslash{}setbeamercolor}\NormalTok{\{block title alerted\}\{bg=\}}
  \FunctionTok{\textbackslash{}setbeamercolor}\NormalTok{\{block title example\}\{bg=\}}
\NormalTok{\}}
\FunctionTok{\textbackslash{}newcommand}\NormalTok{\{}\ExtensionTok{\textbackslash{}moloch@block@fill}\NormalTok{\}\{}
  \FunctionTok{\textbackslash{}setbeamercolor}\NormalTok{\{block title\}\{}\CommentTok{\%}
\NormalTok{    bg=normal text.bg!80!fg}
\NormalTok{  \}}
  \FunctionTok{\textbackslash{}setbeamercolor}\NormalTok{\{block body\}\{}\CommentTok{\%}
\NormalTok{    use=block title,}
\NormalTok{    bg=block title.bg!50!normal text.bg}
\NormalTok{  \}}
  \FunctionTok{\textbackslash{}setbeamercolor}\NormalTok{\{block title alerted\}\{}\CommentTok{\%}
\NormalTok{    bg=block title.bg,}
\NormalTok{  \}}
  \FunctionTok{\textbackslash{}setbeamercolor}\NormalTok{\{block title example\}\{}\CommentTok{\%}
\NormalTok{    bg=block title.bg,}
\NormalTok{  \}}
\NormalTok{\}}
\FunctionTok{\textbackslash{}setbeamercolor}\NormalTok{\{block title\}\{}\CommentTok{\%}
\NormalTok{  use=normal text,}
\NormalTok{  fg=normal text.fg}
\NormalTok{\}}
\FunctionTok{\textbackslash{}setbeamercolor}\NormalTok{\{block title alerted\}\{}\CommentTok{\%}
\NormalTok{  use=\{block title, alerted text\},}
\NormalTok{  fg=alerted text.fg}
\NormalTok{\}}
\FunctionTok{\textbackslash{}setbeamercolor}\NormalTok{\{block title example\}\{}\CommentTok{\%}
\NormalTok{  use=\{block title, example text\},}
\NormalTok{  fg=example text.fg}
\NormalTok{\}}
\FunctionTok{\textbackslash{}setbeamercolor}\NormalTok{\{block body alerted\}\{use=block body, parent=block body\}}
\FunctionTok{\textbackslash{}setbeamercolor}\NormalTok{\{block body example\}\{use=block body, parent=block body\}}
\end{Highlighting}
\end{Shaded}

Footnotes

\begin{Shaded}
\begin{Highlighting}[]
\FunctionTok{\textbackslash{}setbeamercolor}\NormalTok{\{footnote\}\{fg=normal text.fg!90\}}
\FunctionTok{\textbackslash{}setbeamercolor}\NormalTok{\{footnote mark\}\{fg=.\}}
\end{Highlighting}
\end{Shaded}

We also reset the bibliography colors in order to pick up the
surrounding colors at the time of use. This prevents us having to set
the correct color in normal and standout mode.

\begin{Shaded}
\begin{Highlighting}[]
\FunctionTok{\textbackslash{}setbeamercolor}\NormalTok{\{bibliography entry author\}\{fg=, bg=\}}
\FunctionTok{\textbackslash{}setbeamercolor}\NormalTok{\{bibliography entry title\}\{fg=, bg=\}}
\FunctionTok{\textbackslash{}setbeamercolor}\NormalTok{\{bibliography entry location\}\{fg=, bg=\}}
\FunctionTok{\textbackslash{}setbeamercolor}\NormalTok{\{bibliography entry note\}\{fg=, bg=\}}
\end{Highlighting}
\end{Shaded}

\subsection{Process Package Options}\label{process-package-options-4}

\begin{Shaded}
\begin{Highlighting}[]
\FunctionTok{\textbackslash{}moloch@color@setdefaults}
\FunctionTok{\textbackslash{}ProcessPgfPackageOptions}\NormalTok{\{/moloch/color\}}
\end{Highlighting}
\end{Shaded}

\begin{Shaded}
\begin{Highlighting}[]
\FunctionTok{\textbackslash{}mode}\NormalTok{\textless{}all\textgreater{}}
\end{Highlighting}
\end{Shaded}

\chapter{Color Theme: Tomorrow}\label{color-theme-tomorrow}

Register tomorrow-specific light and dark color schemes

\begin{Shaded}
\begin{Highlighting}[]
\FunctionTok{\textbackslash{}mode}\NormalTok{\textless{}all\textgreater{}}
\end{Highlighting}
\end{Shaded}

\chapter{Color Theme: High Contrast}\label{color-theme-high-contrast}

Register high-contrast-specific light and dark color schemes

\begin{Shaded}
\begin{Highlighting}[]
\FunctionTok{\textbackslash{}mode}\NormalTok{\textless{}all\textgreater{}}
\end{Highlighting}
\end{Shaded}





\end{document}
