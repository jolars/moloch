\documentclass{beamer}

% Parse colortheme and variant from \jobname
% Expected format: colortheme-<theme>-<variant>
% Example: colortheme-catppuccin-dark
% If not in that format, use defaults
\def\parsejobname#1-#2-#3\relax{%
  \def\themename{#2}%
  \def\variantname{#3}%
}
\expandafter\parsejobname\jobname-default-light\relax

% Check if parsing failed (no hyphen in \jobname)
% If \themename contains the full \jobname, use defaults
\edef\checkname{\jobname-default}
\ifx\themename\checkname
  \def\themename{default}%
  \def\variantname{light}%
\fi

% Load theme with parsed options
\usetheme[colortheme=\themename, colortheme variant=\variantname]{moloch}

\title{Moloch Color Theme Example}

\date{\today}
\author{Johan Larsson}
\institute{Some University}

\begin{document}

\maketitle

\begin{frame}
  \frametitle{Color Theme Showcase}

  This example demonstrates the color theme with different block
  styles\footnote{Theme: \themename, Variant: \variantname}.

  \begin{columns}[T]
    \begin{column}{0.45\textwidth}
      \begin{block}{Default}
        Block content.
      \end{block}

      \begin{alertblock}{Alert}
        Block content.
      \end{alertblock}

      \begin{exampleblock}{Example}
        Block content.
      \end{exampleblock}
    \end{column}
    \begin{column}{0.45\textwidth}
      {
        \molochset{block=fill}

        \begin{block}{Default}
          Block content.
        \end{block}

        \begin{alertblock}{Alert}
          Block content.
        \end{alertblock}

        \begin{exampleblock}{Example}
          Block content.
        \end{exampleblock}
      }
    \end{column}
  \end{columns}
\end{frame}

\section{A Section Page}

\begin{frame}[standout]
  Questions?
\end{frame}

\molochset{progressbar=frametitle}

\begin{frame}
  \frametitle{Progress Bars}

  This slide demonstrates the progress bar below the frame title.
\end{frame}

\end{document}
